\documentclass[12pt]{article}
\usepackage[T1]{fontenc}
\usepackage{ebgaramond}
\let\bfseries\mdseries
\usepackage[margin=1in]{geometry}
\usepackage{semantic-markup}
\usepackage{microtype}
\frenchspacing
\usepackage[all]{nowidow}
\usepackage[notes]{biblatex-chicago}
\addbibresource{master.bib}
\setcounter{secnumdepth}{0}

\title{Request for Junior Faculty Leave, Fall 2022}
\author{Andrew A. Cashner}

\makeatletter
\RenewDocumentCommand{\maketitle}{}{%
    \begin{center}
        \tabular{c}
        {\Large\@title} \\[1.5em]
        {\large\@author} \\[1em]
        {\large\@date}
        \endtabular
    \end{center}
    \bigskip
}
\makeatother

\begin{document}
\maketitle

I am requesting a Junior Faculty Leave for the fall semester of 2022 to
support my research project entitled \wtitle{Songs at the Woods’ Edge: The
Earth Songs of the Seneca Nation}.
This is be a collaborative project with Bill Crouse, Sr., the renowned
Seneca master musician and faithkeeper, focused on the traditional repertoire
of Onöndowa'ga:' (Seneca) social dance songs, also known as earth songs.

We envision two main outputs: (1) a co-authored book of public scholarship and
(2) a website featuring high-quality videos of performances of the songs and
dances and teaching about them.
Both the book and the website will provide the first in-depth, reliable
guide to this vital aspect of Haudenosaunee (Iroquois) culture for scholars,
educators, and the public, presented from an indigenous perspective and
according to indigenous values.  
The book will focus on these social dance songs as a living tradition, drawing
on Bill Crouse’s expertise as a practitioner of the oral tradition; and as a
historic one, drawing on my archival research in Lewis Henry Morgan’s papers
and other sources that document these songs in the earliest intercultural
encounters. 

\section{Book Outline}

The central concept of the book comes from the historic Seneca practice of the
\quoted{Woods'-Edge Greeting}, which was a way of welcoming outsiders into the
village and played an important role both in the functioning of the
Haudenosaunee League government and in treaty negotiations with
Euro-Americans.%
\Autocite{Richter:Ordeal}
Seneca musicians are still using these songs to create a meeting space for
building reciprocal relationships between Native and non-Native people, as
well as helping Seneca people reconnect with their own heritage.
In fact, the entryway to Ganondagan, the Seneca Cultural Center near Victor,
is designed around the traditional Woods'-Edge Greeting, and Bill Crouse
regularly presents social songs there in events that are free and open to the
public, using the songs as a way to teach traditional Onöndowa'ga values and
to highlight the resiliency and vibrancy of Seneca culture today.
We intend for the book and website, likewise, to stand in the clearing at the
woods' edge, using the earth songs to welcome anyone who wants to build a
closer relationship to the original people of this land and their traditions.

We envision five chapters for the book, with the first three based primarily
on interviews with Bill Crouse and the last two based more on my historical
research.
The first chapter will focus on the songs' relationship to the earth:
we will explore the geographical places and environments in which the songs
are performed today, and where they were sung across the ancestral
Onöndowa'ga:' territory currently occupied by western New York.
This chapter will also pass on traditional teachings about the spiritual
dimension of the songs as means of connecting people to the earth and to each
other.
The second chapter will go into detail about the musical structure and
patterns of these songs, not imposing European analytical categories but
helping readers understand indigenous ways of understanding singing and dance.
The third chapter will discuss teaching and tradition, tracing genealogies of
teachers and methods of oral transmission, and showing how the Seneca people
kept these traditions alive despite many obstacles, from land dispossession
and forced removal to boarding schools and the Kinzua dam tragedy.
Moreover it will show how the younger generations of Onöndowa'ga:' musicians
are still responding creatively to tradition and finding its relevance to
their contemporary situation.

Chapter four will trace the origin and history of the songs, connecting
indigenous oral traditions with written archival documents that make it
possible to trace these songs back to the first encounters with Europeans.
This chapter will correct errors and fill in gaps in the small body of
existing scholarship, and draw on previously unrecognized archival sources,
seen through indigenous eyes.
The final chapter will look at the earth songs within the context of a long
history of intercultural exchange, preceding European encounter and continuing
today, in which Haudenosaunee people have used songs at the woods' edge to
share their community with outsiders and build relationships with them based
on mutual benefit.

\section{Context and Significance}

This project would meet a significant need for trustworthy information about
Native American music in our region, provided with the authority and blessing
of indigenous experts and their communities.
Beverley Diamond's introduction to Haudenosaunee music is sound but very
brief, and it models the collaborative approach between non-Native scholars
Native experts that we will be using.%
\Autocite{Diamond:NativeAmericanNortheast}
We hope that our work will be able to correct the serious errors and
shortcomings in the few more detailed works of published scholarship on this
topic.
The first detailed description of Seneca social dances was written by Lewis
Henry Morgan in his \wtitle{The League of the Haudenosaunee, or Iroquois},
published here in Rochester in 1851, and based on his first-hand observations
of Seneca practices with the help of Seneca informants.%
\Autocite{Morgan:League}
He describes dances as so central to Haudenosaunee culture that \quoted{they
contain within themselves a picture and a realization of Indian life}, to the
extent that when the dance \quoted{loses its attractions, they will cease to
be Indians} (263).%
\Autocite[261, 263]{Morgan:League}
He later argues, however, that Native Americans will only be able to survive
in \quoted{civilization} if they give up their traditional ways, and so his
book is at once a tribute to indigenous culture and an instrument of its
destruction.
Morgan's papers are preserved here at the University of Rochester, and despite
his bias, the records contain some of the oldest detailed information about
Seneca dance.
Like many Euro-American observers before him, he observed many practices that
he did not understand, but his descriptions may yield intelligible information
to contemporary pracitioners like Bill Crouse.

In the twentieth century, William Fenton and Gertrude Kurath published
ethnographic studies of Haudenosaunee dances and ceremonial songs which,
though based on first-hand observation, were not sufficiently informed by
insider perspectives or attuned to the priorities and values of the people
they were observing.%
\Autocites{Fenton:GreatLaw}{Kurath:IroquoisMusic}
Kurath is the only source for detailed study of the actual sound and structure
of Haudenosaunee songs, but her taxonomies and speculative analysis are based
on criteria foreign to indigenous teachings.
In Fenton's case, his work is actually considered by many Seneca people to be
both offensive and factually wrong.
He made ceremonial information public that was meant to be kept private, even
within Haudenosaunee communities.
Though his attempt to reconstruct the original ceremonies by which the
Haudenosaunee League constituted itself was ultimately unsuccessful, Fenton
nevertheless identified a large number of historical and archival sources that
describe songs and dances still practiced today.
As with Morgan, then, Fenton gathered much valuable information even if he
made questionable use of it.
Scrutiny of his published and manuscript writings together with an indigenous
expert will make it possible to correct the published scholarly record and is
likely to yield information about Seneca traditions that may have been lost
over the years and could be of value to the Seneca Nation.
In this sense the historical portion of this project will be an act of
intellectual and cultural repatriation.

\section{Funding}

I would use the money first to film a series of interviews and performance
videos at significant sites around ancestral Seneca territory: funds will
cover fees for Mr. Crouse and other performers, professional audio and visual
technicians and equipment, and travel, so that we can present the songs in the
most appealing and engaging way we can.
I will also need to travel funding for archival research: Fenton's papers
and numerous historic audio recordings of Seneca earth songs are at the
American Philosophical Society in Philadelphia, and there are relevant
holdings at the New York State Museum in Albany and the archives of the
University of Buffalo and Syracuse University, as well as institutions in
Canada.
I have the technical skills to build the website myself but we will need
funding for domain registration, hosting, and maintenance.
For the book we may need funding for image permissions and copy editing.

I am applying for different funding sources for distinct components of the
project. 
For the website and book, I am applying for an NEH-Mellon Fellowship for
Digital Publication, which offers a maximum award amount of \$5,000 per month
for six to twelve months, starting January 1, 2023 at the earliest.
I also intend to apply for a Digital Justice Grant from the American Council
of Learned Societies; the Seed Grant program offers up to \$25,000 for a
period of twelve to eighteen months, starting July 1, 2022, at the earliest.
For filming the social dance songs, I will apply for a grant from the National
Endowment for the Arts; grants range from \$10,000 to \$100,000 and would
start in January 2023. 
For travel to archives and reproduction of materials, especially historic
field recordings of Seneca songs, I am also applying for smaller travel grants
from the Phillips Fund for Native American Research of the American
Philosophical Society (average \$3,000, awarded in May 2022) and from the
Library of Congress, American Folklife Center.

\section{Timeline}

The project's timeline depends on funding.
An ideal timeline would be to film in the summer of 2022 and put together the
website and book during the 2022--2023 academic year.
My archival and historical research is ongoing, and I have already begun
initial interviews and teaching sessions with Bill Crouse.
Making professional videos in significant locations will obviously require
more funding; if sufficient funds are available this summer we will film then,
though some videos may need to be filmed at different times for their seasonal
significance.
Much of the project can continue independently of the filming, however.
Whenever the funding comes, it seems feasible to have a finished product by
the end of calendar year 2023.

Even though some funding sources will not be available until 2023, I am
requesting the leave in fall 2022 for several reasons.
First, my tenure review will begin in fall 2022, and therefore this leave
would provide me with an opportunity to strengthen my tenure case by making
significant progress toward a second book.
I had originally planned to take an extra year before tenure due to pandemic
hardships, but now that things have settled down my chair supported my
request go up for tenure this year.
Even with the compressed timeline I would benefit from the opportunity to use
the junior leave while I can still help advance my tenure portfolio.

Second, I have designed two new courses that are offered in the spring
semester and and I think it would serve the department best if I could
continue teaching them next year.
These courses are Arranging, a completely new course which is successfully
attracting members of the student \term{a capella} and theater groups into the
department, and History of Western Music 1600--1800, which I have redesigned
to bring it up to date with contemporary scholarship and respond to ethical
concerns about diversity and inclusion which we are already trying to address
in the department through a more globally-oriented curriculum.

\section{Rationale and Benefit to the University and Community}

While this research project on Native American music grew out of my existing
work on colonial music and popular religious devotion and I will draw on my
experience with archival research, it also marks a significant shift in
direction.
Even aside from the specific timeline of the research project I have outlined,
a leave will provide me with the time I need to extend into this new area.

This work extends my work on music and religious belief in colonial Latin
America into a new part of the world, requiring a shift in methodology and a
new ethical sensitivity with respect to indigenous peoples.%
\Autocites{Cashner:HearingFaith}{Cashner:Cards}{Cashner:ImitatingAfricans}
I began this project with the support of a Humanities Center Fellowship in
spring 2020. 
Since then I have been developing relationships with people in the Seneca
Nation and learning the Seneca language with Ja:no's Bowen of the Allegany
Territory Language Department.
I have done most of the necessary background reading in primary and secondary
sources and have begun working with archival sources in our own Rare Books and
Special Collections relative to local Native American history.
I have incorporated what I have been learning into my teaching, as I have
included units on Haudenosaunee song in courses on music appreciation and
music history, including hosting Bill Crouse as a guest teacher in spring 2021
and again this April.

The opportunity to pivot and broaden my intellectual horizons will keep my
research and teaching fresh and responsive to the concerns of our age. 
I hope to emerge from the leave not only with \wtitle{Songs from the Woods'
Edge} drafted or even completed, but also with plans for new course units or
an entirely new course (such as Music in Early America), and new relationships
with indigenous musicians who can contribute to the university community as
guest teachers, presenters, and performers.
On campus I hope to draw on the resources of our Department of Audio and Music
Engineering to produce the media, and I can envision fruitful collaborations
with our programs in dance and theater and events through the Humanities
Center to present our findings to our academic community and to the public.
In the outside community I will seek connections with Ganondagan and the
Seneca-Iroquois National Museum in Salamanca.
In these ways I can help the university become a leader in building restorative
relationships with Native American communities, in developing new approaches
to teaching and scholarship that integrate Native perspectives and are truly
relevant to the lives of people in our surrounding community.
\end{document}
