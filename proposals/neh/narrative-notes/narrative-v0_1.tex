\documentclass{neh}
\begin{document}
\section{Narrative}
\subsection{Significance and Contribution}
The original inhabitants of the land now occupied by western New York are the
members of the Onöndowa’ga:’ or Seneca Nation, one of the Six Nations of the
Haudenosaunee (Iroquois) Confederacy.
The traditional body of social dance songs known as Earth Songs still form a
central part of community life for traditional Senecas on the territories of
the Seneca Nation of Indians, the Tonawanda Band of Senecas, and at Six
Nations Ontario.
In contrast to longhouse ceremonies, which are closed to non-Senecas and some
of which are even restricted within the nation to those who have need of them,
the Earth Songs are shared openly with the public, and have served for
centuries as ways of building reciprocal relationships with non-indigenous
people.
The earliest European interlopers in Seneca country report being greeted at
the woods' edge with songs, and seeing whole commmunities united in song and
dance.
At Ganondagan, the Seneca Arts and Culture Center near Victor, New York,
visitors pass through an entryway designed around the traditional woods'-edge
greeting to hear regular presentations of Earth Songs by Seneca musicians like
the Salamanca-based faithkeeper, ritualist, and artist Bill Crouse, Sr.
There Mr. Crouse and other singers use the Earth Songs to create a space like
the woods'-edge clearing of earlier days in which to share Seneca teachings
and values with outsiders.

This project is a collaboration with Bill Crouse, with the goal of presenting
Seneca music to the academic community and general public for the first time
accurately, sensitively, and on Seneca terms.
We envision a digital humanities project that presents high-quality videos of
the songs and dances with indigenous teaching and historical research about
the songs' origins and significance, as part of a co-authored, born-digital
book of public scholarship.
The book will focus on these social dance songs as a living tradition, drawing
on Bill Crouse's experetise as a practitioner of the oral tradition; and as as
historic one, drawing on my archival research in the papers of anthropologists
Lewis Henry Morgan (who first described the dances in English) and William
Fenton, and records of Seneca performances in the region through the
period of colonization.

The clearing at the woods' edge stands symbolically at the boundary between
indigenous traditional knowledge and modern experience under colonization,
between Seneca communities and Euro-American ones.
As Bill Crouse explains, the songs connect participants with Mother Earth
(\foreign{Ethinö’eh yöëdzade’}), uniting the community with all living and
non-living beings.
As an ancient oral tradition that practitioners are constantly finding new
ways to employ to meet present needs, the Earth Songs also connect history and
tradition, memory and creativity.
This project is the first academic study dedicated to Seneca Earth Songs, and
the first study of Seneca music and dance to focus on the perspectives of
indigenous practitioners.

\subsubsection{Relationship to Existing Studies}

This study aims to address a lack of trustworthy, in-depth resources for
non-indigenous people who are interested in Native American music, who want to
build a closer relationship to the Seneca people, and who are seeking a deeper
understanding of the role of indigenous people in American history.
Apart from presentations at places like Ganondagan, it is difficult for
scholars, teachers, and community members to know whether they can trust the
handful of online videos or the outdated scholarly publications.
The first detailed description of Seneca social dances was written by Lewis
Henry Morgan in his \wtitle{The League of the Haudenosaunee, or Iroquois},
published here in Rochester in 1851, and based on his first-hand observations
of Seneca practices with the help of Seneca informants
\Autocite{Morgan:League}.
He describes dances as so central to Haudenosaunee culture that \quoted{they
contain within themselves a picture and a realization of Indian life}, to the
extent that when the dance \quoted{loses its attractions, they will cease to
be Indians}
\Autocite[261, 263]{Morgan:League}.
He later argues, however, that Native Americans will only be able to survive
in \quoted{civilization} if they give up their traditional ways, and so his
book is at once a tribute to indigenous culture and an instrument of its
destruction.

After Morgan there only a few published studies that investigate Haudenosaunee
social songs, made by anthropologist William Fenton and dance ethnographer
Gertrude Kurath in the 1940s--60s
\Autocites{FentonKurath:EagleDance}{Kurath:IroquoisMusic}{Caldwell:Kurath}.
Kurath is the only source for detailed study of the actual sound and structure
of Haudenosaunee songs, but her taxonomies and speculative analysis are based
on criteria foreign to indigenous teachings; Fenton's work is considered by
many Seneca people to be both offensive and factually wrong
\Autocite{McCarthy:Iroquoianist}.
It is in large part because of the cavalier and inaccurate way with which
these researchers treated longhouse ceremony that those ceremonies are now
kept so strictly private.
In other words, the only available scholarly resources that specifically treat
Seneca social dance do so in the context of a large amount of information that
Seneca faithkeepers do not want the public to see or hear, partly for their
own protection.

The proposed project differs from these previous studies because it focuses on
music that Seneca people are actually willing to share, and moreover, on music
that Seneca people are currently using to build intercultural relationships.
This project will highlight the voice---both singing and teaching---of an
expert Seneca musician as well as perspectives from other musicians, and will
aim to present their views in their own terms without importing foreign
analytical categories or using their lived experience as raw material for
scholarly theorization in the exploitative fashion that has characterized
previous attempts.
This approach is in line with more recent approaches that model collaborative
work between non-Native scholars and Native experts, such as Beverley
Diamond's excellent but brief introduction to Haudenosaunee music 
\Autocite{Diamond:NativeAmericanNortheast}; 
and with a growing literature that emphasizes the modernity and creativity of
Native music as a contemporary practice
\Autocites
{Browner:FirstNations}
{Browner:Heartbeat}
{LevineRobinson:MusicModernity}.

\subsubsection{Benefit to Scholars and the Public}

This project will benefit humanities scholars who need reliable information
about Haudenosaunee traditions not only to understand indigenous values in
their own right but also to properly understand American history.
Too much past scholarship on Native music presented indigenous peoples in a
historical present tense: they told the story about \quoted{how Indians sing}
with the assumption that their singing like their broader identity is part
of the past and will eventually remain there.
Morgan argued that unlike the unrelentingly progressive and expansive aspect
of Western civilization, Indian culture was static and could never adapt to
the modern world.
Though Fenton and Kurath do not explicitly share Morgan's racialized ideology
of white supremacy, their accounts still treat present-day Native practice as
static and of primary interest for what it tells white researchers about the
past.
What is worse, no textbook history of Western music used in today's collegiate
curricula includes Native American music as part of America's story, past or
present, except perhaps for a brief mention in connection with Dvo\v{r}ak's
\wtitle{New World Symphony}.
Early-American music scholarship still rarely integrates indigenous knowledge,
which would have strengthened otherwise excellent recent historical studies of
colonial American music drawn from Euro-American documents
\Autocites{Goodman:IndianPsalmody}{Eyerly:Moravian}.
In fact even those documents reveal the constant presence of Native song in
the sound world of early America, from the earliest Jesuit Relations from
French Canada to the eighteenth-century accounts of English treaty proceedings
printed by Benjamin Franklin, from Morgan's observations in the 1840s through
annual public performances in the 1930s at Seneca encampments in Rochester's
Maplewood Park.
Seneca social songs have been performed in downtown Rochester for far longer
than Handel's \wtitle{Messiah} have been sung by the Rochester Oratorio
Society or Beethoven's symphonies have been played at the Eastman School of
Music; because, defying Morgan's prophecy, Seneca people have never stopped
living here on their ancestral lands, and they have never ceased dancing.

By collaboratively combining archive-based historical work and ethnographic
research, including looking together at archival sources with indigenous
experts, this project will provide a more balanced view of the sounds of early
America and the way music functioned in the intercultural dynamics that
shaped this country.
Moreover, compared to the staid rituals of a conservatory recital, in which
standard programs have changed very little in a century, a semi-annual
\quoted{Sing} event could include hundreds of newly composed songs (in the
\foreign{ehsgä:nye’} genre), not to mention musicians like the late Joanne
Shenandoah who fuse traditional and modern styles.

\subsubsection{Rationale for Digital Publication}

The interactive, interlinked nature of a website is well suited to the
relational and participatory character of the Earth Songs and the way they are
taught and shared in Seneca communities.
The digital format will allow the book-website to be freely and readily
accessible to a much wider audience than most academic music books reach, and
in an open format that is more amenable to input and modification from the
Seneca community and more in keeping with traditional Haudenosaunee values for
how to share and preserve cultural knowledge.
The project will demonstrate that one cannot understand the history of music
in America without knowing Haudenosaunee song, and that Native song is neither 
stuck in a primitive present tense nor lost to the past.
On the contrary, the website will highlight the vibrancy and power of
contemporary Seneca dancing and singing as a way of furthering the cause of
indigenous self-determination and cultural revitalization.
The recovery of descriptions, song lists, and recordings from the ethnographic
papers of Morgan and Fenton (among other archival sources) will constitute a
form of intellectual and cultural repatriation
\Autocite{Fox:Repatriation}.

\subsection{Organization, Concepts, and Methods}

The key concepts in this project are three pairs of terms: Earth/land,
relationship/reciprocity, and tradition/history.
Seneca social songs both celebrate and enact a spiritual and ecological
relationship with the Earth, while also connecting Seneca people to the
specific land of their ancestry.
While traditional teaching about the songs centers the ecological aspects of
the connection to the Earth, filming the songs in significant places will
boster the spiritual, moral, and legal claims of the Seneca nation to their
homelands (guaranteed to them by the still-valid Candandaigua Treaty of 1794)
\Autocites{Deloria:BrokenTreaties}{BasicCall}, and the historical aspect of
the project will firmly situate Seneca singing as an ever-present part of the
history of this region even from a Euro-American perspective.

Relationship and reciprocity are widely acknowledged core values for Native
North Americans, and for the Haudenosaunee historically and today these
concepts are central to everything from how songs are passed down, how they
are presented, and how they function in the community and as bridge to those
outside. 
The concept of the Covenant Chain recurs throughout
Haudenosaunee--Euro-American treaty negotiations: an iron chain linking the
first European ship that arrived on the North Atlantic coast to the notional
longhouse of the Haudenosaunee confederacy, which must be kept clean and
polished free of rust by both parties.
This project aims to resume the long-overdue work (from the American side) of
polishing the covenant chain, restoring the mutually beneficial relationships
originally proposed by the Haudenosaunee to the newcomers from overseas.

Finally, moving dialectically between tradition and history enables a better
understanding Native music-making than the primitivistic notions of the
anthropologists who viewed indigenous song as not having a history and not
being part of history.
At the same time the goal is not simply to fit Native music into an existing
Westernized historical framework; for indigenous North Americans, singing
itself constitutes a form of historical knowledge and provides its own ways of
connecting past, present, and future
\Autocite{Diamond:NativeAmericanHistory}.

\subsubsection{Components}

The website will include a general introduction and pages for each of the main
body of the songs, including videos, recordings, teaching, and historical
information.
The songs will be recorded in beautiful and significant outdoor locations
across ancestral Seneca territory, in different seasons and conditions
depending on the meanings of the particular song.
Pages on individual songs will include notated transcriptions, stories about
how the songs were learned or taught, and philosophical reflections on their
connection to Seneca traditional worldview and ethics.
The website will also provide information about issues of cultural
sensitivity, appropriation, and ethical use.

Every precaution will be taken to ensure that the site only shares material
acceptable to representatives of the Seneca Nation.
I will form a panel of consultants to test and advise about the site, which
will include indigenous musicians, scholars, and community members as well as
other subject-matter experts.
Ideally even the technology can be structured in a way that harmonizes with
Seneca values, as some recent projects have demonstrated
\Autocite{Christen:RelationshipsNotRecords}.
One of the chief benefits of the site to Seneca people, according to Bill
Crouse, would be to make accessible, all in location, a full library of
historic social-song recordings---those made by ethnographers like Fenton and
those captured on cassette tape by participants in a Sing.
I will consult with indigenous contributors to ensure that all materials on
the site are made available with appropriate licenses.
\Autocite{Christen:RelationshipsNotRecords}.

Though the precise relationship of book and website is still developing, we
envision five chapters, with the first three based primarily on interviews
with Bill Crouse and the last two based more on my historical research.
The first chapter will focus on the songs' relationship to the earth:
we will explore the geographical places and environments in which the songs
are performed today, and where they were sung across the ancestral
Onöndowa’ga:’ territory.
This chapter will also pass on traditional teachings about the spiritual
dimension of the songs as means of connecting people to the earth and to each
other.
The second chapter will go into detail about the musical structure and
patterns of these songs, not imposing European analytical categories but
helping readers understand indigenous ways of understanding singing and dance.
The third chapter will discuss teaching and tradition, tracing genealogies of
teachers and methods of oral transmission, and showing how the Seneca people
kept these traditions alive despite many obstacles, from land dispossession
and forced removal to boarding schools and the Kinzua dam tragedy.
Moreover it will show how the younger generations of Onöndowa'ga:' musicians
are still responding creatively to tradition and finding its relevance to
their contemporary situation.

Chapter four will trace the origin and history of the songs, connecting
indigenous oral traditions with written archival documents that make it
possible to trace these songs back to the first encounters with Europeans.
This chapter will correct errors and fill in gaps in the small body of
existing scholarship, and draw on previously unrecognized archival sources,
seen through indigenous eyes.
The final chapter will look at the earth songs within the context of a long
history of intercultural exchange, preceding European encounter and continuing
today, in which Haudenosaunee people have used songs at the woods' edge to
share their community with outsiders and build relationships with them based
on mutual benefit.

\subsubsection{Sources}

There are several kinds of sources for the project.
First there are the contemporary performances of Earth Songs by Bill Crouse
and other Seneca singers and dancers, which I will observe and record in
various locations as described above.
In addition to the Seneca Nation of Indians with its Allegany and Cattaraugus
territories I would also like to draw in the perspectives of practitioners in
the other Seneca polities, the Tonawanda Band of Senecas and the Seneca at Six
Nations in Ontario.

There is also a sizeable corpus of recorded Earth Songs made by
anthropologists, the majority housed today at the American Philosophical
Society (APS) in Philadelphia and the Library of Congress, and most of them
available digitally.
There are also commercial recordings by the Allegany Singers and other groups,
as well as personal recordings, many on cassette tape, made by community
members.
Given the problems with the published studies, to gain understanding of the
functions and significance of the songs it will be necessary to conduct
interviews with Bill Crouse and other Seneca experts and community members.
Some of these interviews will be recorded on video and can be used directly on
the site (that is, not needing to pass through the filter of me as the
ethnographer). 

For the historical portions, Lewis Henry Morgan's papers are housed at my own
institution, the University of Rochester, while his collection is part of the
larger holdings of Haudenosaunee objects at the Rochester Museum and Science
Center (RMSC); Fenton's papers, which include detailed, unpublished notes on
Seneca dance, are at the APS.
There are photographs of Seneca musicians and dances in the above archives as
well as that of the RMSC, the New York State Museum, and others.


\subsection{Competencies, Skills, and Access}

This is a new area of research for me.
My previous work focused on music in religious devotion, with a master's
thesis (2009) on hymn singing in German Lutheran communities, then a
dissertation (2015) and monograph (2020) on devotional music and theology in
the Spanish Empire of the seventeenth century. 
That work drew me more ever more into colonial and decolonial studies to
explore processes of cultural exchange at the frontiers of the early modern
European world, while out of a desire to better understand my own adopted home
in Rochester I began to pursue the role of song in intercultural encounters in
early America, with the support of a Humanities Center Fellowship from the
University of Rochester in 2020.
I quickly discovered that I had far more to learn about the indigenous side of
those encounters, and that existing work in that area answered few of my
questions, in part because of the problems noted above.
The more I learned about Haudenosaunee history and culture, and the more
relationships I built with Seneca people, the more I realized that before
going any further I would need to understand Seneca music, past and present,
and that although there are traces in historical documents of Seneca songs,
they can only be understood from the perspective of the oral tradition
practiced today.
Through an initial invitation to teach in my Music History classroom, I
have been building a relationship with Bill Crouse and working with him to
develop a project that would be of mutual benefit and meet Seneca community
standards for cultural sensitivity.

My qualifications for this project, new as it is, stem on the one hand from my 
academic experience in archive-based historical research, and on the other
from my professional experience as a Christian church musician, in which I am
a practitioner of an oral tradition and keeper of tradition.
These are distinct and incommensurable traditions, of course, but I do
know first-hand how song traditions are cultivated within a community, and my
graduate training did include ethnomusicology and ritual studies.
Because fieldwork and interviews are new research methods for me, I sought
out mentorship from the esteemed ethnomusicologist Ellen Koskoff.
I am also seeking out the counsel of indigenous scholars particularly on
ethical issues.

As a musician I have ample experience learning and transcribing traditional
songs by ear, and I have a strong facility for languages.
I have been learning Seneca from Ja:no’s Bowen of the Seneca Nation Language
Department for two years.
Learning the language to an intermediate beginner level has been vital for
understanding aspects of worldview and culture, but full proficiency will take
a long time for an outsider to a language with less than thirty fluent
speakers.
It is a help to the novice that the words of the Earth Songs are primarily
vocables (fixed but meaningless sound combinations).
I demonstrated my technological proficiency for digital humanities by
typesetting two volumes of critical music editions in the \LaTeX{}
document-preparation system and the Lilypond music-notation system, which I
also used to create all the examples, tables, maps, and diagrams for my
mongraph with Brill; and by building my research website and class websites
for every course I have taught since 2017 (\url{www.andrewcashner.com}).
I demonstrated my software engineering abilities in creating a digital
implementation of a device for automatic computing from 1650 and making it
available through a web application (\url{www.arca1650.info}).
Having worked as a copy editor for seven years, I copy-edited my own monograph
and am prepared to ensure the highest standards in this project as well.

\subsection{Final Product and Dissemination}

The website itself will be built with sustainable technology, using little
scripting and if possible no fancy plugins that can so easily break or become
obsolete (as anyone still trying to use a Flash-based site can testify).
It will be accessible to people using screen readers and will be equally
useable and attractive on desktop and mobile devices.
The sites I have built before may not have trendy animations or elaborate
interactive features, but because they are built on the official standards of
the Internet itself (HTML5, CSS), they will never break or become obsolete.
I know how to use adaptive design to create a site that works equally well
for desktop or mobile viewing and is fully accessible to people using
text-based browsers and screen readers, and a print publication made to the
highest typographical standards---both from the same sources.

Using open-source software is essential for a project that aims to serve an
economically under-privileged community; in fact these tools serve as
\quoted{liberation technologies} that contribute to the cause of indigenous
self-determination, not dependent on the arcane processes of traditional
publishers or the whims of territorial peer reviewers
\autocite[\XXX]{BasicCall}.
It also makes the publication easier to update, correct, and maintain.
I will be responsible for maintaining the website but I will also seek out an
assistant webmaster, preferably from the Seneca Nation.
I have purchased the domain \url{http://www.senecasongs.info} and have created
a working prototype of the site, though I have not yet shared it publicly
because of cultural sensitivity concerns and in order to allow for peer
review.

Since this project will not go through a publisher I will have to develop my
own system for peer review.
I will assemble a panel of consultants who will do the initial peer review,
and this panel will include both indigenous academics and non-academic
practitioners.
The chair of the panel will be empowered to seek out additional, potentially
anonymous reviews as well.
The panel will hold to a tight schedule.
I will also ask Seneca nation members to test the website and I will hire
proficient Seneca-language speakers to review the linguistic elements in the
project.

\end{document}
