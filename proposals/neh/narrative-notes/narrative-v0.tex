\documentclass{neh}
\begin{document}
\section{Narrative}
\subsection{Significance and Contribution}
% thesis
% overview: basic ideas, problems, questions
% significance to humanities scholars, general audience, or both
% how it will complement, challenge, or expand relevant existing studies
% why digital publication is essential, how scholarship will be presented to
%   benefit audiences

\wtitle{Songs at the Woods’ Edge: The Earth Songs of the Seneca Nation} is a
collaborative project with Bill Crouse, Sr., the renowned Seneca master
musician and faithkeeper, focused on the traditional repertoire of
Onöndowa'ga:' (Seneca) social dance songs, also known as Earth Songs.
We envision two main outputs: (1) a co-authored book of public scholarship and
(2) a website featuring high-quality videos of performances of the songs and
dances with teaching about them.
Both the book and the website will provide the first in-depth, reliable
guide to this vital aspect of Haudenosaunee (Iroquois) culture for scholars,
educators, and the public, presented from an indigenous perspective and
according to indigenous values.  
The book will focus on these social dance songs as a living tradition, drawing
on Bill Crouse’s expertise as a practitioner of the oral tradition; and as a
historic one, drawing on my archival research in Lewis Henry Morgan’s papers
and other sources that document these songs in the earliest intercultural
encounters. 

The \quoted{Woods'-Edge Greeting} was a way of welcoming outsiders into
the village and played an important role both internally in the political
functioning of the confederacy government and externally, in treaty
negotiations with Euro-Americans
\Autocites{Richter:Ordeal}{Fenton:GreatLaw}.
The Onöndowa’ga:’ (Seneca) Nation, in their position as the Keepers of the
Western Door of the notional longhouse of the Haudenosaunee Confederacy,
used these customs to welcome many kinds of indigenous and Euro-American
wayfarers, missionaries, and traders.
Today members of the Seneca Nation continue to live on their ancestral lands
in the region occupied by the state of New York (including both the Seneca
Nation of Indians and the Tonawanda Band of Senecas), and the clearing at the
woods' edge stands symbolically at the boundary between indigenous traditional
knowledge and modern experience, between Seneca communities and Euro-American
ones, between history and tradition, memory and creativity.

The social dance songs that play this role are known as Earth Songs because
they connect participants to Mother Earth (\foreign{Ethinö’ëh Yöëdzade’}),
uniting the community with all living and non-living beings.
Though traditional longhouse ceremonies today are kept strictly closed to
non-Seneca observers, Seneca singers and dancers use Earth Songs to welcome
outsiders into their circle even as they strengthen their own communities.
One frequent site for Earth Song performances is Ganondagan, the Seneca
Cultural Center near Victor, New York, where the entryway is designed around
the traditional Woods'-Edge Greeting.
Bill Crouse regularly presents social songs there in events that are free and
open to the public, using the songs as a way to teach traditional Onöndowa'ga
values and to highlight the resiliency and vibrancy of Seneca culture today.
We intend for the book and website, likewise, to stand in the clearing at the
woods' edge, using the Earth Songs to welcome anyone who wants to build a
closer relationship to the original people of this land and their traditions.

This project would meet a significant need for trustworthy information about
Native American music in this region, provided with the authority and blessing
of indigenous experts and their communities.
The first detailed description of Seneca social dances was written by Lewis
Henry Morgan in his \wtitle{The League of the Haudenosaunee, or Iroquois},
published here in Rochester in 1851, and based on his first-hand observations
of Seneca practices with the help of Seneca informants
\Autocite{Morgan:League}.
He describes dances as so central to Haudenosaunee culture that \quoted{they
contain within themselves a picture and a realization of Indian life}, to the
extent that when the dance \quoted{loses its attractions, they will cease to
be Indians}
\Autocite[261, 263]{Morgan:League}.
He later argues, however, that Native Americans will only be able to survive
in \quoted{civilization} if they give up their traditional ways, and so his
book is at once a tribute to indigenous culture and an instrument of its
destruction.
Morgan's papers are preserved here at the University of Rochester, and despite
his bias, the records contain some of the oldest detailed information about
Seneca dance.
Like many Euro-American observers before him, he observed many practices that
he did not understand, but his descriptions may yield intelligible information
to contemporary pracitioners like Bill Crouse.

The few scholarly studies of Seneca song that include any detail about sound or
movement were made by white academic observers based on fieldwork and field
recordings in the 1930s--50s; they focused largely on ceremonial songs rather
than social songs and were not adequately informed by the perspectives or
values of indigenous insiders
\Autocites{FentonKurath:EagleDance}{Kurath:IroquoisMusic}.
Kurath is the only source for detailed study of the actual sound and structure
of Haudenosaunee songs, but her taxonomies and speculative analysis are based
on criteria foreign to indigenous teachings
\Autocite{Caldwell:Kurath}.
In Fenton's case, much of his work is actually considered by many Seneca
people to be both offensive and factually wrong
\Autocite{McCarthy:Iroquoianist}.
He made ceremonial information public that was meant to be kept private, even
within Haudenosaunee communities.
Though his attempt to reconstruct the original ceremonies by which the
Haudenosaunee League constituted itself was ultimately unsuccessful, Fenton
nevertheless identified a large number of historical and archival sources that
describe songs and dances still practiced today
\Autocites{Fenton:GreatLaw}.
As with Morgan, then, Fenton gathered much valuable information even if he
made questionable use of it.
Scrutiny of his published and manuscript writings together with an indigenous
expert will make it possible to correct the published scholarly record and is
likely to yield information about Seneca traditions that may have been lost
over the years and could be of value to the Seneca Nation.
In this sense the historical portion of this project will be an act of
intellectual and cultural repatriation.

The proposed project differs from these previous studies because it focuses on
music that Seneca people are actually willing to share, and moreover, on music
that Seneca people are currently using to build intercultural relationships.
This project will highlight the voice---both singing and teaching---of an
expert Seneca musician as well as perspectives from other musicians, and will
aim to present their views in their own terms without importing foreign
analytical categories or using their lived experience as raw material for
scholarly theorization in the exploitative fashion that has characterized
previous attempts.
This approach is in line with more recent approaches that model collaborative
work between non-Native scholars and Native experts, such as Beverley
Diamond's excellent but brief introduction to Haudenosaunee music 
\Autocite{Diamond:NativeAmericanNortheast}; 
and with a growing literature that emphasizes the modernity and creativity of
Native music as a contemporary practice
\Autocites
{Browner:FirstNations}
{Browner:Heartbeat}
{LevineRobinson:MusicModernity}.

Teachers at the collegiate and secondary level who would like to go introduce
students to Haudenosaunee music in way that goes beyond superficial or
stereotyped portrayals.
Because the studies of Fenton and Kurath do not demarcate privileged
ceremonial information, it is not even possible for a scholar or educator to
page through the books without violating Seneca protocols, and even the
portions on social songs are rife with inaccurate and perjorative
generalizations (e.g., that most Indian melodies, like other remnants of
primitive culture, start on a higher note and go down to a lower note).

This project will benefit humanities scholars who need reliable information
about Haudenosaunee traditions not only to understand indigenous values in
their own right but also to properly understand American history.
Too much past scholarship on Native music presented indigenous peoples in a
historical present tense: they told the story about \quoted{how Indians sing}
with the assumption that their singing like their broader identity is part
of the past and will eventually remain there.
Morgan argued that unlike the unrelentingly progressive and expansive aspect
of Western civilization, Indian culture was static and could never adapt to
the modern world. 
Shorn of Morgan's overt racialized ideology of white supremacy, Fenton and
Kurath's accounts still treat present-day Native practice as static and of
primary interest for what it tells white researchers about the past.
What is worse, no textbook history of Western music used in today's collegiate
curricula includes Native American music as part of America's story, past or
present, except perhaps for a brief mention in connection with Dvo\v{r}ak's
\wtitle{New World Symphony}.
Early-American music scholarship still rarely integrates indigenous knowledge,
which would have strengthened otherwise excellent recent historical studies of
colonial American music drawn from Euro-American documents
\Autocites{Goodman:IndianPsalmody}{Eyerly:Moravian}.

In fact even those documents reveal the constant presence of Native song in
the sound world of early America, from the earliest Jesuit Relations from
French Canada to the eighteenth-century accounts of English treaty proceedings
printed by Benjamin Franklin, from Morgan's observations in the 1840s through
annual public performances in the 1930s at Seneca encampments in Rochester's
Maplewood Park.
Seneca social songs have been performed in downtown Rochester for far longer
than Handel's \wtitle{Messiah} have been sung by the Rochester Oratorio
Society or Beethoven's symphonies have been played at the Eastman School of
Music; because, defying Morgan's prophecy, Seneca people have never stopped
living here on their ancestral lands, and they have never ceased dancing.


By collaboratively combining archive-based historical work and ethnographic
research, including looking together at archival sources with indigenous
experts, this project will provide a more balanced view of the sounds of early
America and the way music functioned in the intercultural dynamics that
shaped this country.
The goal is not simply to fit Native music into an existing Westernized
historical framework, however; for indigenous North Americans, singing itself
constitutes a form of historical knowledge and provides its own ways of
connecting past, present, and future
\Autocite{Diamond:NativeAmericanHistory}.
Moreover, compared to the staid rituals of a conservatory recital, in which
standard programs have changed very little in a century, a semi-annual
\quoted{Sing} event could include hundreds of newly composed songs (in the
\foreign{ehsgä:nye’} genre), not to mention musicians like the late Joanne
Shenandoah who fuse traditional and modern styles---showcasing the vitality
and creativity of Haudenosaunee singing today.

 
\subsection{Organization, Concepts, and Methods}
% how research will resolve problems I am examining
% theoretical framework, and how this research will advance it
% explain concepts, terminology
% sources

% current state of project, which stage will be supported by fellowship
% digital methods, reasons for choosing particular technology

% outline of chapters and digital publication's design
% data I will generate or collect
% how I will address public access, privacy, confidentiality, security,
%   intellectual property, rights

\subsubsection{Sources}

There are several kinds of sources for the project.
First there are the contemporary performances of Earth Songs by Bill Crouse
and other Seneca singers and dancers, which I will observe and record in
various locations as described above.
In addition to the Seneca Nation of Indians with its Allegany and Cattaraugus
territories I would also like to draw in the perspectives of practitioners in
the other Seneca polities, the Tonawanda Band of Senecas and the Seneca at Six
Nations in Ontario.

There is also a sizeable corpus of recorded Earth Songs made by
anthropologists, the majority housed today at the American Philosophical
Society (APS) in Philadelphia and the Library of Congress, and most of them
available digitally.
There are also commercial recordings by the Allegany Singers and other groups,
as well as personal recordings, many on cassette tape, made by community
members.
Given the problems with the published studies, to gain understanding of the
functions and significance of the songs it will be necessary to conduct
interviews with Bill Crouse and other Seneca experts and community members.
Some of these interviews will be recorded on video and can be used directly on
the site (that is, not needing to pass through the filter of me as the
ethnographer). 

For the historical portions, Lewis Henry Morgan's papers are housed at my own
institution, the University of Rochester, while his collection is part of the
larger holdings of Haudenosaunee objects at the Rochester Museum and Science
Center (RMSC); Fenton's papers, which include detailed, unpublished notes on
Seneca dance, are at the APS.
There are photographs of Seneca musicians and dances in the above archives as
well as that of the RMSC, the New York State Museum, and others.

\subsubsection{Website}

The website will include video and audio recordings of the core repertoire of
Earth Songs as practiced by Bill Crouse, with videos and articles about the
technique, significance, origins, and history of the songs.
The videos will be recorded at historically and spiritually important
locations throughout ancestral Seneca territory, including \foreign{Ohi:yo:’}
(the Seneca Nation's territory on the Allegheny River), the massive waterfalls
of Letchworth State Park, the site of the largest Seneca village of the
seventeenth century at Ganondagan (Victor, NY), the historical trail (now
Indian Trail Road) in Rochester's Mount Hope Cemetery.
They will be recorded in different seasons and conditions depending on the
meanings of the particular song.
Pages on individual songs will include notated transcriptions, stories about
how the songs were learned or taught, and philosophical reflections on their
connection to Seneca traditional worldview and ethics.
The website will also provide information about issues of cultural
sensitivity, appropriation, and ethical use.

Every precaution will be taken to ensure that the site only shares material
acceptable to representatives of the Seneca Nation.
I will form a panel of consultants to test and advise about the site, which
will include indigenous musicians, scholars, and community members as well as
other subject-matter experts.
Ideally even the technology can be structured in a way that harmonizes with
Seneca values, as some recent projects have demonstrated
\Autocite{Christen:RelationshipsNotRecords}.
One of the chief benefits of the site to Seneca people, according to Bill
Crouse, would be to make accessible, all in location, a full library of
historic social-song recordings---those made by ethnographers like Fenton and
those captured on cassette tape by participants in a Sing.
This would effectively repatriate the ethnographic recordings and contribute
to Seneca cultural and linguistic revival efforts
\Autocite{Fox:Repatriation}.
I will consult with indigenous contributors to ensure that all materials on
the site are made available with appropriate licenses.

\subsubsection{Book}

For the book, we envision five chapters, with the first three based primarily
on interviews with Bill Crouse and the last two based more on my historical
research.
The first chapter will focus on the songs' relationship to the earth:
we will explore the geographical places and environments in which the songs
are performed today, and where they were sung across the ancestral
Onöndowa’ga:’ territory.
This chapter will also pass on traditional teachings about the spiritual
dimension of the songs as means of connecting people to the earth and to each
other.
The second chapter will go into detail about the musical structure and
patterns of these songs, not imposing European analytical categories but
helping readers understand indigenous ways of understanding singing and dance.
The third chapter will discuss teaching and tradition, tracing genealogies of
teachers and methods of oral transmission, and showing how the Seneca people
kept these traditions alive despite many obstacles, from land dispossession
and forced removal to boarding schools and the Kinzua dam tragedy.
Moreover it will show how the younger generations of Onöndowa'ga:' musicians
are still responding creatively to tradition and finding its relevance to
their contemporary situation.

Chapter four will trace the origin and history of the songs, connecting
indigenous oral traditions with written archival documents that make it
possible to trace these songs back to the first encounters with Europeans.
This chapter will correct errors and fill in gaps in the small body of
existing scholarship, and draw on previously unrecognized archival sources,
seen through indigenous eyes.
The final chapter will look at the earth songs within the context of a long
history of intercultural exchange, preceding European encounter and continuing
today, in which Haudenosaunee people have used songs at the woods' edge to
share their community with outsiders and build relationships with them based
on mutual benefit.

\subsection{Competencies, Skills, and Access}
% my competence and background in area of project
% if new, reasons for working in it and qualifications to do so
% self-assessment of expertise in digital humanities & technologies involved
% any institutional support
% level of competence in relevant language
% where study will be conducted, what research materials
% describe access to archives, collections, institutions

% Need to address ethnomusicology disciplinary question

This project grew out of my previous research on colonial music and the way
people use music not only to express religious beliefs, but to embody
their fundamental worldview, with focus on the Spanish Empire of the
seventeenth century.
This work extends my work on music and religious belief in colonial Latin
America into a new part of the world, requiring a shift in methodology and a
new ethical sensitivity with respect to indigenous peoples.

My prior research was based on research in nine archives in the United States,
Mexico, and Spain, as well as a large amount of early modern books and
manuscripts available online.
I am also a highly competent musician, having worked as a church musician in
some capacity almost continuously since 2003.
I am skilled at learning new music by ear and notating it, and I understand
music from the practical standpoint of the practitioner while also considering
it critically as a scholar.

My language skills are strong: I speak Spanish and German and can read Latin,
Greek, most of the Romance languages and several Germanic languages.
I have been learning Seneca with Ja:no’s Bowen of the Seneca Nation Language
Department since fall 2019.
With not more than thirty fluent speakers living, and the intercultural
challenges of approaching the Seneca community as a white academic outsider,
it is difficult to be immersed in the language, but I have achieved an
intermediate beginner skill level in this difficult language.

I am bringing all of these experiences and skills to bear as I move in this
direction for my research. 
I was motivated initially by the desire to look at intercultural encounters in
colonial America, but quickly discovered that existing accounts either looked
only at Euro-American documentary sources or were heavily filtered through the
bias of mid-twentieth-century ethnographers, as noted above.
In the wake of George Floyd's murder, as I marched in Black Lives Matter
protests here in Rochester in response to the murder of Daniel Prude by
police, in conversation with other musicologists I came to see that the way
most of us had been studying and teaching music was equivalent to Southern
plantation tours that leave out discussion of slavery to focus on the
beautiful artworks that survive in the master's house.
As I began to read works by indigenous writers I became more and more interest
in the history and significance of the land on which I live.
As a Midwesterner briefly dislocated to California and now transplanted here,
whose research had focused on the beliefs of people centuries ago in distant
countries, I wanted to do work that would help me understand where I live.
Moreover, I wanted to do work that might actually contribute to the redress of
wrongs against indigenous people perpetrated by my own ancestors and their
compatriots.
Rather than spend my career writing books and articles that essentially
constitute long-form program notes for Classical music concerts, or moving
further into theoretical discourses in early modern studies and historical
sound studies that work to insulate researchers from confronting ethical
responsibilites in the present, I wanted to correct and supplement what had
been missing in my own education, and what was missing in my own teaching, and
contribute something that would actually help both indigenous and
non-indigenous people reckon with this country's past and find ways to build a
better future.

I have sought out mentorship in ethnomusicology from Ellen Koskoff, one of the
foremost practitioners of that discipline, and I will continue to build a
network of advisors in this new area.
I would not characterize my research as a shift from historical musicology to
ethnomusicology, however, as in many ways these two disciplines were formed in
opposition to each other.
Using ethnography to answer historical questions, using archival sources to
answer contemporary questions, resisting the extractive use of indigenous
knowledge in the service of academic theorizing while also insisting that
indigenous music-making should inform our understanding of American music
history rather than being put in a ghetto or on a pedestal---all of these
priorities are constantly driving me back and forth across disciplinary
boundaries.

I began this project with the support of a Humanities Center Fellowship from
the University of Rochester in spring 2020, though because of the pandemic
most of the funding was cut and my plans for fieldwork and archival travel had
to be postponed.
Since then I have been developing relationships with people in the Seneca
Nation and learning the Seneca language with Ja:no's Bowen of the Allegany
Territory Language Department.
I have done most of the necessary background reading in primary and secondary
sources and have begun working with archival sources in UR's own Rare Books and
Special Collections relative to local Native American history.
I have incorporated what I have been learning into my teaching, as I have
included units on Haudenosaunee song in courses on music appreciation and
music history, including hosting Bill Crouse as a guest teacher in spring 2021
and again this April.


I would use the money first to film a series of interviews and performance
videos at significant sites around ancestral Seneca territory: funds will
cover fees for Mr. Crouse and other performers, professional audio and visual
technicians and equipment, and travel, so that we can present the songs in the
most appealing and engaging way we can.
I will also need to travel funding for archival research: Fenton's papers
and numerous historic audio recordings of Seneca earth songs are at the
American Philosophical Society in Philadelphia, and there are relevant
holdings at the New York State Museum in Albany and the archives of the
University of Buffalo and Syracuse University, as well as institutions in
Canada.
I have the technical skills to build the website myself but we will need
funding for domain registration, hosting, and maintenance.
For the book we may need funding for image permissions and copy editing.

On campus I hope to draw on the resources of our Department of Audio and Music
Engineering to produce the media, and I can envision fruitful collaborations
with our programs in dance and theater and events through the Humanities
Center to present our findings to our academic community and to the public.
In the outside community I will seek connections with Ganondagan and the
Seneca-Iroquois National Museum in Salamanca.

\subsection{Final Product and Dissemination}
% intended results, how it will benefit audience
% accessibility to people with disabilities
% how it will be evaluated before publication, e.g., peer review
% how, where accessed (e.g., web), when publication will be available and how
%   disseminated. Provide URL of website
% any discussions with publishers?

% plans for maintaining digital project long term, for how long and by whom

The precise format of the book remains to be determined, as well as its
relationship to the website.
One possibility is to write a traditional print book and use the website as a
multimedia companion 
\autocite[as in]{Eyerly:Moravian}.
My previous monograph \wtitle{Hearing Faith} (Leiden: Brill, 2020) was
designed to be used together with critical editions of the music
studied, which were made freely available online at the same time
(\wtitle{Villancicos about Music} vols. 1--2, Web Library of
Seventeenth-Century Music no. 32, 36, \url{http://www.sscm-wlscm.org/}).
This project would work similarly except that the website will complement the
monograph with a much richer array of multimedia sources.

But it would also be possible to create the book \emph{as} a website. 
This would avoid privileging the scholar's perspective to the same degree, and
would emphasize both a plurality of perspectives and the relational and
community-oriented nature of indigenous knowledge.
It makes it possible to present the project as a web of knowledge from
different sources, where users may encounter side by side and with relatively
equal weight, a song recording, an indigenous practitioner's explanation of
that song, and my own historical research on descriptions of the song in
historic accounts and ethnographies. 

The website itself will be built with sustainable technology, using little
scripting and if possible no fancy plugins that can so easily break or become
obsolete (as anyone still trying to use a Flash-based site can testify).
It will be accessible to people using screen readers and will be equally
useable and attractive on desktop and mobile devices.
I have maintained my own website since about 2013 and I have built a special
website for every class I have taught since 2017.
These sites do not have trendy animations or elaborate interactive features,
but because they are built on the official standards of the Internet itself
(HTML5, CSS), they will never break or become obsolete.
I have been learning computer programming since 2012 and am highly experienced
with several kinds of digital typesetting, especially using the \LaTeX{}
document-preparation system, to which I have contributed several widely-used
software packages (and which I used to generate these application documents).
I know how to create both a responsive-design website that adapts to desktop
or mobile viewing and is fully accessible to people using text-based browsers
and screen readers, and a print publication made to the highest typographical
standards---both from the same sources.
I published two volumes of critical music editions using the open-source
systems \LaTeX{} for text and Lilypond for music, and I used these systems to
generate all the bibliography, tables, maps, diagrams, and parallel
translations in my monograph with Brill.

Using open-source software is essential for a project that aims to serve an
economically under-privileged community; in fact these tools serve as
\quoted{liberation technologies} that contribute to the cause of indigenous
self-determination, not dependent on the arcane processes of traditional
publishers or the whims of territorial peer reviewers
\autocite[\XXX]{BasicCall}.
It also makes the publication easier to update, correct, and maintain.
I will be responsible for maintaining the website but I will also seek out an
assistant webmaster, preferably from the Seneca Nation.
I have purchased the domain \url{http://www.senecasongs.info} and have created
a working prototype of the site, though I have not yet shared it publicly
because of cultural sensitivity concerns and in order to allow for peer
review.

In my experience publishing my monograph and journal articles, publishers
today are adding very little value to scholarly work; in these cases, the time
and expense required for publication consisted mostly in correcting errors
introduced by the editors or typesetters themselves, manually fixing problems
for which there are well-established programmatic solutions.
When working with other publishers, as with journal articles, I have had to
first typeset a complete article  myself so that it can pass through peer
review, and then un-typeset it and convert it to primitive Microsoft Word
formats, which the publisher then retypesets using some of the same
technologies as I did originally, but having introduced numerous errors and
inconsistencies.
Copy-editing is at such a low state that, for example, in Diamond's
introduction to Native music of the Northeast
\autocite{Diamond:NativeAmericanNortheast}, 
Oxford University Press swapped cardinal directions in her description of the
Haudenosaunee (putting Senecas at the Eastern door) and mispelled the word
\mentioned{ethnomusicology} on the back cover page.
Peer review has its own problems, and in my experience has been used as much
for gatekeeping and advancing personal agendas as for validating the research
or improving the writing.
It is rarely possible for authors who have been presenting publicly on their
work to be anonymous, while anonymity shields abusive and unethical reviewers.
The review process results in delays such that it took me more than two years
to publish my most recent article.

This project, therefore, will seek a different approach. 
I will assemble a panel of consultants who will do the initial peer review,
and this panel will include not indigenous academics and non-academic
practitioners.
The chair of the panel will be empowered to seek out additional, potentially
anonymous reviews as well.
The panel will hold to a tight schedule.
Having worked as a copy editor for seven years, I copy-edited my own monograph
and am prepared to ensure the highest standards in this project as well,
though I will also ask proficient Seneca-language speakers to review the
linguistic elements in the project.


\end{document}
