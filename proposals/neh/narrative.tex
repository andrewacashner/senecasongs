\documentclass{neh}
\title{Narrative}
\begin{document}
\maketitle
% {{{1 significance
\section{Significance and Contribution}
The original inhabitants of the land now occupied by western New York are the
members of the Onöndowa’ga:’ or Seneca Nation, one of the Six Nations of the
Haudenosaunee (Iroquois) Confederacy.
The traditional body of social dance songs known as Earth Songs still form a
central part of community life for traditional Senecas on the territories of
the Seneca Nation of Indians, the Tonawanda Band of Senecas, and at Six
Nations Ontario.
In contrast to longhouse ceremonies, which are closed to non-Senecas and some
of which are even restricted within the nation to those who have need of them,
the Earth Songs are shared openly with the public, and have served for
centuries as ways of building reciprocal relationships with non-indigenous
people.
The earliest European interlopers in Seneca country report being greeted at
the woods' edge with songs, and seeing whole commmunities united in song and
dance.
At Ganondagan, the Seneca Arts and Culture Center near Victor, New York,
visitors pass through an entryway designed around the traditional woods'-edge
greeting to hear regular presentations of Earth Songs by Seneca musicians like
the Salamanca-based faithkeeper, ritualist, and artist Bill Crouse, Sr.
There Mr. Crouse and other singers use the Earth Songs to create a space like
the woods'-edge clearing of earlier days in which to share Seneca teachings
and values with outsiders.

This project is a collaboration with Bill Crouse, with the goal of presenting
Seneca music to the academic community and general public for the first time
accurately, sensitively, and on Seneca terms.
We envision a digital humanities project that presents high-quality videos of
the songs and dances with indigenous teaching and historical research about
the songs' origins and significance, as part of a co-authored, born-digital
book of public scholarship.
The book will focus on these social dance songs as a living tradition, drawing
on Bill Crouse's experetise as a practitioner of the oral tradition; and as as
historic one, drawing on my archival research in the papers of anthropologists
Lewis Henry Morgan (who first described the dances in English) and William
Fenton, and records of Seneca performances in the region through the
period of colonization.

The clearing at the woods' edge stands symbolically at the boundary between
indigenous traditional knowledge and modern experience under colonization,
between Seneca communities and Euro-American ones.
As Bill Crouse explains, the songs connect participants with Mother Earth
(\foreign{Ethinö’eh yöëdzade’}), uniting the community with all living and
non-living beings.
As an ancient oral tradition that practitioners are constantly finding new
ways to employ to meet present needs, the Earth Songs also connect history and
tradition, memory and creativity.
This project is the first academic study dedicated to Seneca Earth Songs, and
the first study of Seneca music and dance to focus on the perspectives of
indigenous practitioners.

\subsection{Relationship to Existing Studies}

This study aims to address a lack of trustworthy, in-depth resources for
learning about this type of Native American music.
According to Peter Jemison, recently retired director of Ganondagan, the
cultural center staff were flooded in the last two years by requests from
educators for information and counsel on how to include Native culture and
history in their curricula.
For Haudenosaunee and Seneca music, there are few reliable sources available.
Materials by indigenous creators can be hard to find and the published
scholarship has serious problems.

In the first detailed description of Seneca social dances (published here in
Rochester), Lewis Henry Morgan describes dances as so central to Haudenosaunee
culture that \quoted{they contain within themselves a picture and a
realization of Indian life}, to the extent that when the dance \quoted{loses
its attractions, they will cease to be Indians}
\Autocite[261, 263]{Morgan:League}.
He later argues that Native Americans will only be able to survive in
\quoted{civilization} if they give up their traditional ways, and so his book
is at once a tribute to indigenous culture and an instrument of its
destruction.
After Morgan the only specific studies on Haudenosaunee song were made in the
mid-twentieth century by William Fenton and Gertrude Kurath, and only Kurath
treats the sound and movements in any detail
\Autocites{FentonKurath:EagleDance}{Kurath:IroquoisMusic}{Caldwell:Kurath}.
These ethnographers published information on longhouse ceremony that community
members now want to be private, and they imported foreign criteria to their
speculative taxonomies and interpretations; as a result many Haudenosaunee
people consider Fenton's work in particular to be both offensive and
inaccurate
\Autocite{McCarthy:Iroquoianist}.
At present, then, an outsider who wishes to learn about Seneca social dance
from these publications cannot avoid being exposed to a large amount of
material that Seneca faithkeepers do not want the public to see or hear,
partly for their own protection.

The proposed project differs from these previous studies because it focuses on
music that Seneca people are actually willing to share, and builds on the way
they are already using this music to build intercultural relationships.
The methodology follows the model of recent collaborative work between
non-Native scholars and Native experts, such as Beverley Diamond's excellent
though brief introduction to Haudenosaunee music 
\Autocite{Diamond:NativeAmericanNortheast},
and with a growing literature that emphasizes the modernity and creativity of
Native music as a contemporary practice
\Autocites
{Browner:FirstNations}
{Browner:Heartbeat}
{LevineRobinson:MusicModernity}.

\subsection{Benefit to Scholars and the Public}

This project will benefit humanities scholars, educators, and members of the
public by providing them with reliable information on Native American
expressive culture.
The knowledge and perspectives shared through this project will help all of
us, whether our ancestors are indigenous, settlers, or enslaved people, to
gain a deeper understanding of the land we all share.
Some indigenous people may deepen their connection to their own traditions;
non-indigenous people will be better equipped to build relationships with
Native American communities.
Historical studies of music in America today, whether in scholarship or in the
college classroom, still largely ignore Native music; and even the best recent
studies that focus on interactions between Native and Euro-American people in
early America still do not incorporate traditional knowledge of indigenous
communities today
\Autocites{Goodman:IndianPsalmody}{Eyerly:Moravian}.
But even Euro-American archival documents reveal the constant presence of
Native song in the sound world of early America, so that no history of
American music can claim coherence without including the music of indigenous
Americans.

The interactive, interlinked nature of a website is well suited to the
relational and participatory character of the Earth Songs and the way they are
taught and shared in Seneca communities.
The digital format will allow the book/website to be freely 
accessible to a wide public audience.
By collaboratively combining archive-based historical work and ethnographic
research, including looking together at archival sources with indigenous
experts, this project will open new insights into the sounds of early America
and provide better resources for building a more just society for all
Americans.

% }}}1
% {{{1 organization, concepts
\clearpage
\section{Organization, Concepts, and Methods}

% {{{2 concepts
The key concepts in this project are three pairs of terms: Earth/land,
relationship/reciprocity, and tradition/history.
Seneca social songs celebrate and enact a relationship with the Earth in both
ecological and spiritual terms, while also connecting Seneca people to the
land of their ancestry.
Filming the songs in important locations in western New York will boster
the Seneca Nation's claims to their homelands, guaranteed to them by
the still-valid Candandaigua Treaty of 1794
\Autocites{Deloria:BrokenTreaties}{BasicCall}; the historical research will
make clear that Seneca singing has always been part of the story of this land.

Relationship and reciprocity are widely acknowledged core values for Native
North Americans, and they define the way Haudenosaunee people teach and
present songs.
The concept of the Covenant Chain recurs throughout colonial-era treaty
negotiations: an iron chain linked the first European ship on the North
Atlantic coast to the longhouse of the Haudenosaunee confederacy, and both
sides had an obligation to keep it free from rust.
For me as an Indiana native descended from German settler-colonialists, this
project provides a way to take up the long-overdue work of polishing the
covenant chain, working toward restoration of mutually beneficial
relationships between indigenous and settler Americans. 

Finally, this project explores the complex relationship between history and
tradition in both indigenous and Western conceptions.
The project will demonstrate that one cannot understand the history of music
in America without knowing Haudenosaunee song, and that Native song is neither 
stuck in a primitive present tense nor lost to the past.
The goal, however, is not simply to fit Native music into a Western historical
framework; 
for indigenous North Americans, singing itself constitutes a form of
historical knowledge and provides its own ways of connecting past, present,
and future
\Autocite{Diamond:NativeAmericanHistory}.
% }}}2
% {{{2 components
\subsection{Components}

The website will feature new high-quality videos of Bill Crouse and others
singing Earth Songs in beautiful and significant outdoor locations
across ancestral Seneca territory.
For each type of song there will be a written introduction, video interviews
or stories about structure and use of the songs, and philosophical reflections
on their relation to Seneca worldview.
The website will also provide users with information about issues of cultural
sensitivity, appropriation, and ethical use; and I will consult with
indigenous contributors to ensure that all materials are made available with
appropriate licenses
\Autocite{Christen:RelationshipsNotRecords}.
Sources include contemporary performances, interviews, and fieldwork
observations; ethnographic recordings at the American Philosophical Society
(APS) in Philadelphia and the Library of Congress; and archival
documents at the University of Rochester (Lewis Henry Morgan papers), the
Rochester Museum and Science Center (Morgan and Ely Parker collections), the
APS (Fenton papers), and the New York State Museum.
One of the chief benefits of the site to Seneca people, according to Bill
Crouse, would be to make accessible in one location a full library of
historic recordings, effectively repatriating the ethnographers' materials 
\Autocite{Fox:Repatriation}.

The book will include five chapters, with the first three based primarily on
interviews with Bill Crouse and the last two based more on my historical
research.
The first chapter will focus on the songs' relationship to the earth and the
land; the second will explore the musical structure and patterns of these
songs, emphasizing Seneca understandings of music.
The third chapter will trace genealogies of teaching and methods of oral
transmission, showing how the Seneca people kept their songs alive in defiance
of land dispossession, boarding schools, and the Kinzua dam tragedy, and how
younger generations are still responding creatively to tradition including
through the COVID-19 pandemic.
Chapter four will trace the origin and history of the songs, connecting
indigenous oral traditions with written archival documents from Euro-American
perspectives, reading Morgan and Fenton's field notes together with indigenous
practitioners.
The final chapter will look at the earth songs within the context of a long
history of intercultural exchange, preceding European encounter and continuing
today, in which Haudenosaunee people have used songs at the woods' edge to
share their community with outsiders and build mutually beneficial
relationships.
% }}}2
% }}}1
% {{{1 competencies
\section{Competencies, Skills, and Access}

I began exploring this new research area with the support of a Humanities
Center Fellowship from my university in 2020, with the goal of understanding
patterns of intercultural exchange among different communities of colonial-era
New York.
This interest grew out of my previous studies of sacred song in early modern
Germany Lutheran and Spanish Catholic communities, which led me into colonial
and decolonial studies.
While funding and travel were limited during the pandemic I did extensive
background reading in Native American studies, history, and ethnomusicology,
and began research in UR's own archive.
Since inviting Bill Crouse to teach in my Music History classroom in 2021 we
have been working together to develop a project that would benefit the Seneca
community.

In this project I will draw on my experience in archive-based historical
research while extending into the new research methods of ethnographic
fieldwork.
While my graduate studies did include ethnomusicology and ritual studies, I have
sought out mentorship from ethnomusicologist Ellen Koskoff, and am
building a network of indigenous advisers.
As a professional musician I have ample experience learning traditional songs
by ear.
After two years of Seneca language study with Ja:no’s Bowen of the Seneca
Nation Language Department I have an intermediate novice ability; for a
language with fewer than thirty fluent speakers and a song repertoire that
uses mostly wordless vocables, fluency is not necessary.

My experience with digital humanities technologies prepares to ensure the
highest standards of typography and design, and to build an efficient and
stable software system.
I worked for seven years as a copy editor, I typeset and published two volumes
of critical music editions in \LaTeX{} and Lilypond, and I preparing the full
text of my monograph in \LaTeX{} with all diagrams, tables, and music
examples.
I have designed and maintained websites for research projects
(\url{www.arca1650.info}) and teaching (\url{www.andrewcashner.com}) since
2012.
My institution has provided \$5,000 of research funding that I can use as
start-up funding, and I will draw on the expertise of faculty and students in
audio-music engineering and computer-science departments.
% }}}2
% }}}1
% {{{1 final product
\section{Final Product and Dissemination}

The final product will some form of born-digital book and website
(\url{http://www.senecasongs.info}).
Readers may experience the book as an online, potentially non-linear,
experience, or as a more traditional book, including the ability to download
the book as a PDF.
The book and website will be generated from the same source texts, using a
sustainable workflow based on free and open-source technology.
The website will be built on the core standards of the Internet (HTML5 and
CSS3) that are supported in every browser, minimizing scripts and plugins that
can break or become obsolete; the print form of the book will be typeset with
\LaTeX{}, a robust system with strong backwards compatibility.
Through adaptive design the site will be equally accessible and ADA-compliant
on desktop and mobile devices and via screen readers.

I will host and maintain the website, but I will also seek out a Seneca
webmaster and provide ways for the community to update or correct the site. 
Using open-source software not only makes the publication easier to mantain,
but it is essential when serving an economically under-privileged community.
These tools provide the \quoted{liberation technologies} demanded by
Haudenosaunee activists and make it possible for Native nations to exercise
their rights, validated by the UN Declaration on the Rights of Indigenous
Peoples, to control their own representation and cultural heritage
\autocite{BasicCall}.

Given the collaborative nature of a project focused on protected indigenous
cultural heritage, the review process must necessarily be distinct from the
traditional academic model.
I plan to assemble a panel of consultants, including both indigenous academics
and non-academic experts.
After the their own initial peer review, the panel will be empowered to seek
out additional, potentially anonymous reviews.
I will hire proficient Seneca-language speakers to review the linguistic
elements and I will invite Seneca community members to test the site.
% }}}1
\end{document}
