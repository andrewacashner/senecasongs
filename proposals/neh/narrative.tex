\documentclass{neh}
\begin{document}
\section{Narrative}
\subsection{Significance and Contribution}
% thesis
% overview: basic ideas, problems, questions
% significance to humanities scholars, general audience, or both
% how it will complement, challenge, or expand relevant existing studies
% why digital publication is essential, how scholarship will be presented to
%   benefit audiences

\wtitle{Songs at the Woods’ Edge: The Earth Songs of the Seneca Nation} is a
collaborative project with Bill Crouse, Sr., the renowned Seneca master
musician and faithkeeper, focused on the traditional repertoire of
Onöndowa'ga:' (Seneca) social dance songs, also known as Earth Songs.
We envision two main outputs: (1) a co-authored book of public scholarship and
(2) a website featuring high-quality videos of performances of the songs and
dances with teaching about them.
Both the book and the website will provide the first in-depth, reliable
guide to this vital aspect of Haudenosaunee (Iroquois) culture for scholars,
educators, and the public, presented from an indigenous perspective and
according to indigenous values.  
The book will focus on these social dance songs as a living tradition, drawing
on Bill Crouse’s expertise as a practitioner of the oral tradition; and as a
historic one, drawing on my archival research in Lewis Henry Morgan’s papers
and other sources that document these songs in the earliest intercultural
encounters. 

The \quoted{Woods'-Edge Greeting} was a way of welcoming outsiders into
the village and played an important role both internally in the political
functioning of the confederacy government and externally, in treaty
negotiations with Euro-Americans
\Autocites{Richter:Ordeal}{Fenton:GreatLaw}.
The Onöndowa’ga:’ (Seneca) Nation, in their position as the Keepers of the
Western Door of the notional longhouse of the Haudenosaunee Confederacy,
used these customs to welcome many kinds of indigenous and Euro-American
wayfarers, missionaries, and traders.
Today members of the Seneca Nation continue to live on their ancestral lands
in the region occupied by the state of New York (including both the Seneca
Nation of Indians and the Tonawanda Band of Senecas), and the clearing at the
woods' edge stands symbolically at the boundary between indigenous traditional
knowledge and modern experience, between Seneca communities and Euro-American
ones, between history and tradition, memory and creativity.

The social dance songs that play this role are known as Earth Songs because
they connect participants to Mother Earth (\foreign{Ethinö’ëh Yöëdzade’}),
uniting the community with all living and non-living beings.
Though traditional longhouse ceremonies today are kept strictly closed to
non-Seneca observers, Seneca singers and dancers use Earth Songs to welcome
outsiders into their circle even as they strengthen their own communities.
One frequent site for Earth Song performances is Ganondagan, the Seneca
Cultural Center near Victor, New York, where the entryway is designed around
the traditional Woods'-Edge Greeting.
Bill Crouse regularly presents social songs there in events that are free and
open to the public, using the songs as a way to teach traditional Onöndowa'ga
values and to highlight the resiliency and vibrancy of Seneca culture today.
We intend for the book and website, likewise, to stand in the clearing at the
woods' edge, using the Earth Songs to welcome anyone who wants to build a
closer relationship to the original people of this land and their traditions.

This project would meet a significant need for trustworthy information about
Native American music in this region, provided with the authority and blessing
of indigenous experts and their communities.
The first detailed description of Seneca social dances was written by Lewis
Henry Morgan in his \wtitle{The League of the Haudenosaunee, or Iroquois},
published here in Rochester in 1851, and based on his first-hand observations
of Seneca practices with the help of Seneca informants
\Autocite{Morgan:League}.
He describes dances as so central to Haudenosaunee culture that \quoted{they
contain within themselves a picture and a realization of Indian life}, to the
extent that when the dance \quoted{loses its attractions, they will cease to
be Indians}
\Autocite[261, 263]{Morgan:League}.
He later argues, however, that Native Americans will only be able to survive
in \quoted{civilization} if they give up their traditional ways, and so his
book is at once a tribute to indigenous culture and an instrument of its
destruction.
Morgan's papers are preserved here at the University of Rochester, and despite
his bias, the records contain some of the oldest detailed information about
Seneca dance.
Like many Euro-American observers before him, he observed many practices that
he did not understand, but his descriptions may yield intelligible information
to contemporary pracitioners like Bill Crouse.

The few scholarly studies of Seneca song that include any detail about sound or
movement were made by white academic observers based on fieldwork and field
recordings in the 1930s--50s; they focused largely on ceremonial songs rather
than social songs and were not adequately informed by the perspectives or
values of indigenous insiders
\Autocites{FentonKurath:EagleDance}{Kurath:IroquoisMusic}.
Kurath is the only source for detailed study of the actual sound and structure
of Haudenosaunee songs, but her taxonomies and speculative analysis are based
on criteria foreign to indigenous teachings
\Autocite{Caldwell:Kurath}.
In Fenton's case, much of his work is actually considered by many Seneca
people to be both offensive and factually wrong
\Autocite{McCarthy:Iroquoianist}.
He made ceremonial information public that was meant to be kept private, even
within Haudenosaunee communities.
Though his attempt to reconstruct the original ceremonies by which the
Haudenosaunee League constituted itself was ultimately unsuccessful, Fenton
nevertheless identified a large number of historical and archival sources that
describe songs and dances still practiced today
\Autocites{Fenton:GreatLaw}.
As with Morgan, then, Fenton gathered much valuable information even if he
made questionable use of it.
Scrutiny of his published and manuscript writings together with an indigenous
expert will make it possible to correct the published scholarly record and is
likely to yield information about Seneca traditions that may have been lost
over the years and could be of value to the Seneca Nation.
In this sense the historical portion of this project will be an act of
intellectual and cultural repatriation.

The proposed project differs from these previous studies because it focuses on
music that Seneca people are actually willing to share, and moreover, on music
that Seneca people are currently using to build intercultural relationships.
This project will highlight the voice---both singing and teaching---of an
expert Seneca musician as well as perspectives from other musicians, and will
aim to present their views in their own terms without importing foreign
analytical categories or using their lived experience as raw material for
scholarly theorization in the exploitative fashion that has characterized
previous attempts.
This approach is in line with more recent approaches that model collaborative
work between non-Native scholars and Native experts, such as Beverley
Diamond's excellent but brief introduction to Haudenosaunee music 
\Autocite{Diamond:NativeAmericanNortheast}; 
and with a growing literature that emphasizes the modernity and creativity of
Native music as a contemporary practice
\Autocites
{Browner:FirstNations}
{Browner:Heartbeat}
{LevineRobinson:MusicModernity}.

This project will benefit humanities scholars who need reliable information
about Haudenosaunee traditions not only to understand indigenous values in
their own right but also to properly understand American history.
Teachers at the collegiate and secondary level who would like to go introduce
students to Haudenosaunee music in way that goes beyond superficial or
stereotyped portrayals.
Too much past scholarship on Native music presented indigenous peoples in a
historical present tense: they told the story about \quoted{how Indians sing}
with the assumption that their singing like their broader identity is part
of the past and will eventually remain there.
Morgan argued that unlike the unrelentingly progressive and expansive aspect
of Western civilization, Indian culture was static and could never adapt to
the modern world. 
Shorn of Morgan's overt racialized ideology of white supremacy, Fenton and
Kurath's accounts still treat present-day Native practice as static and of
primary interest for what it tells white researchers about the past.
What is worse, no textbook history of Western music used in today's collegiate
curricula includes Native American music as part of America's story, past or
present, except perhaps for a brief mention in connection with Dvo\v{r}ak's
\wtitle{New World Symphony}.
Early-American music scholarship still rarely integrates indigenous knowledge,
which would have strengthened otherwise excellent recent historical studies of
colonial American music drawn from Euro-American documents
\Autocites{Goodman:IndianPsalmody}{Eyerly:Moravian}.

In fact even those documents reveal the constant presence of Native song in
the sound world of early America, from the earliest Jesuit Relations from
French Canada to the eighteenth-century accounts of English treaty proceedings
printed by Benjamin Franklin, from Morgan's observations in the 1840s through
annual public performances in the 1930s at Seneca encampments in Rochester's
Maplewood Park.
Seneca social songs have been performed in downtown Rochester for far longer
than Handel's \wtitle{Messiah} have been sung by the Rochester Oratorio
Society or Beethoven's symphonies have been played at the Eastman School of
Music; because, defying Morgan's prophecy, Seneca people have never stopped
living here on their ancestral lands, and they have never ceased dancing.
By collaboratively combining archive-based historical work and ethnographic
research, including looking together at archival sources with indigenous
experts, this project will provide a more balanced view of the sounds of early
America and the way music functioned in the intercultural dynamics that
shaped this country.


% motivations:
%     - decolonization, reappropriation of land and resources
%     - indigenous self-determination: promote indigenous traditional culture
%     both to outside world and within indigenous communities
%     - educate: provide trustworthy information in place of ignorance,
%     misconceptions, stereotypes
%     - trace the history of the tradition through recordings, descriptions
%     going back to Morgan and even to 17C accounts; show that the music has a
%     history
%     - write Haudenosaunee music into the history of American music;
%     demonstrate persistence and continuity through colonization and today;
%     connect these traditions to others; put them on the same map, part of same
%     story of music on this land
%     - on the other hand, tell the story of Haudenosaunee music from an
%     indigenous perspective, especially its connections to the land and the
%     whole belief system/way of life of traditional indigenous people. Redraw
%     the map using native lifeways; reassert the dominance of indigenous
%     sovereignty over the land and its music
%     - highlight ways that the tradition is continuing, evolving, changing,
%     moving into future; connections to other indigneous music-making (e.g.,
%     intertribal pow-wow)
%     - celebrate what Seneca people love about this music, why it is important
%     to them (including aesthetics, social aspects, core values and beliefs,
%     identity)
%     - elevate the values of the Great Law and the
%     political/ethical/theological principles of Haudenosaunee peoples as
%     embodied in this tradition
% 
%     - provide material useful for educators, secondary and post-secondary
%     - show how indigenous music can be included appropriately in an
%     anticolonial curriculum
%     - create opportunities for Bill Crouse and other Seneca musicians
%     (financial, reputational)
%     - polish the covenant chain; follow the roots to the tree of peace
%
%    - LAND
%    - RELATIONSHIP, COMMUNITY
%    - TRADITION, HISTORY
%
% challenges, problems
%   - taboo on sharing ceremonial information; past abuse by scholars
%   - dangers of colonial thinking in "integrating" ("one map") approach
%   - how to talk about tradition without putting Indians in either the
%   eternal present tense or the dead past?
%   - cultural sensitivity concerns with sharing media
%   - ethnographic challenge of representing communities scattered in several
%   polities with differing attitudes
  
\subsection{Organization, Concepts, and Methods}
% how research will resolve problems I am examining
% theoretical framework, and how this research will advance it
% explain concepts, terminology
% sources

% current state of project, which stage will be supported by fellowship
% digital methods, reasons for choosing particular technology

% outline of chapters and digital publication's design
% data I will generate or collect
% how I will address public access, privacy, confidentiality, security,
%   intellectual property, rights

% THEORY: indigenous theory based in land, reciprocity, relationship,
% connection
% concepts of history and tradition (and history OF tradition)
% problems of recording, notating, sharing
% problems of methods of teaching (oral tradition, relationship-based, access
% to information controlled)

% *** perhaps the project should partly be an inquiry into the right way to do the
% project
% - maybe better not to act like I've resolved all the questions now

We envision five chapters for the book, with the first three based primarily
on interviews with Bill Crouse and the last two based more on my historical
research.
The first chapter will focus on the songs' relationship to the earth:
we will explore the geographical places and environments in which the songs
are performed today, and where they were sung across the ancestral
Onöndowa'ga:' territory currently occupied by western New York.
This chapter will also pass on traditional teachings about the spiritual
dimension of the songs as means of connecting people to the earth and to each
other.
The second chapter will go into detail about the musical structure and
patterns of these songs, not imposing European analytical categories but
helping readers understand indigenous ways of understanding singing and dance.
The third chapter will discuss teaching and tradition, tracing genealogies of
teachers and methods of oral transmission, and showing how the Seneca people
kept these traditions alive despite many obstacles, from land dispossession
and forced removal to boarding schools and the Kinzua dam tragedy.
Moreover it will show how the younger generations of Onöndowa'ga:' musicians
are still responding creatively to tradition and finding its relevance to
their contemporary situation.

Chapter four will trace the origin and history of the songs, connecting
indigenous oral traditions with written archival documents that make it
possible to trace these songs back to the first encounters with Europeans.
This chapter will correct errors and fill in gaps in the small body of
existing scholarship, and draw on previously unrecognized archival sources,
seen through indigenous eyes.
The final chapter will look at the earth songs within the context of a long
history of intercultural exchange, preceding European encounter and continuing
today, in which Haudenosaunee people have used songs at the woods' edge to
share their community with outsiders and build relationships with them based
on mutual benefit.

\Autocites
%{Christen:Mukurtu}
{Christen:RelationshipsNotRecords}
{Fox:Repatriation}
{Diamond:NativeAmericanHistory}

\subsection{Competencies, Skills, and Access}
% my competence and background in area of project
% if new, reasons for working in it and qualifications to do so
% self-assessment of expertise in digital humanities & technologies involved
% any institutional support
% level of competence in relevant language
% where study will be conducted, what research materials
% describe access to archives, collections, institutions

% Need to address ethnomusicology disciplinary question

While this research project on Native American music grew out of my existing
work on colonial music and popular religious devotion and I will draw on my
experience with archival research, it also marks a significant shift in
direction.
Even aside from the specific timeline of the research project I have outlined,
a leave will provide me with the time I need to extend into this new area.

This work extends my work on music and religious belief in colonial Latin
America into a new part of the world, requiring a shift in methodology and a
new ethical sensitivity with respect to indigenous peoples.
I began this project with the support of a Humanities Center Fellowship in
spring 2020. 
Since then I have been developing relationships with people in the Seneca
Nation and learning the Seneca language with Ja:no's Bowen of the Allegany
Territory Language Department.
I have done most of the necessary background reading in primary and secondary
sources and have begun working with archival sources in our own Rare Books and
Special Collections relative to local Native American history.
I have incorporated what I have been learning into my teaching, as I have
included units on Haudenosaunee song in courses on music appreciation and
music history, including hosting Bill Crouse as a guest teacher in spring 2021
and again this April.


I would use the money first to film a series of interviews and performance
videos at significant sites around ancestral Seneca territory: funds will
cover fees for Mr. Crouse and other performers, professional audio and visual
technicians and equipment, and travel, so that we can present the songs in the
most appealing and engaging way we can.
I will also need to travel funding for archival research: Fenton's papers
and numerous historic audio recordings of Seneca earth songs are at the
American Philosophical Society in Philadelphia, and there are relevant
holdings at the New York State Museum in Albany and the archives of the
University of Buffalo and Syracuse University, as well as institutions in
Canada.
I have the technical skills to build the website myself but we will need
funding for domain registration, hosting, and maintenance.
For the book we may need funding for image permissions and copy editing.

On campus I hope to draw on the resources of our Department of Audio and Music
Engineering to produce the media, and I can envision fruitful collaborations
with our programs in dance and theater and events through the Humanities
Center to present our findings to our academic community and to the public.
In the outside community I will seek connections with Ganondagan and the
Seneca-Iroquois National Museum in Salamanca.

\subsection{Final Product and Dissemination}
% intended results, how it will benefit audience
% accessibility to people with disabilities
% how it will be evaluated before publication, e.g., peer review
% how, where accessed (e.g., web), when publication will be available and how
%   disseminated. Provide URL of website
% any discussions with publishers?

% plans for maintaining digital project long term, for how long and by whom

In these ways I can help the university become a leader in building restorative
relationships with Native American communities, in developing new approaches
to teaching and scholarship that integrate Native perspectives and are truly
relevant to the lives of people in our surrounding community.


\end{document}
