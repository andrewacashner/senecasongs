\documentclass{neh}
\title{Narrative}
\begin{document}
\maketitle
% {{{1 significance
\section{Significance and Contribution}
The original inhabitants of the land now occupied by western New York are the
members of the Onöndowa’ga:’ or Seneca Nation, one of the Six Nations of the
Haudenosaunee (Iroquois) Confederacy.
The traditional body of social dance songs known as Earth Songs still form a
central part of community life for traditional Senecas on the territories of
the Seneca Nation of Indians, the Tonawanda Band of Senecas, and at Six
Nations Ontario.
In contrast to longhouse ceremonies, which are closed to non-Senecas and some
of which are even restricted within the nation to those who have need of them,
the Earth Songs are shared openly with the public, and have served for
centuries as ways of building reciprocal relationships with non-indigenous
people.
The earliest European interlopers in Seneca country report being greeted at
the woods' edge with songs, and seeing whole commmunities united in song and
dance.
At Ganondagan, the Seneca Arts and Culture Center near Victor, New York,
visitors pass through an entryway designed around the traditional woods'-edge
greeting to hear regular presentations of Earth Songs by Seneca musicians like
the Salamanca-based faithkeeper, ritualist, and artist Bill Crouse, Sr.
There Mr. Crouse and other singers use the Earth Songs to create a space like
the woods'-edge clearing of earlier days in which to share Seneca teachings
and values with outsiders.

This project is a collaboration with Bill Crouse, with the goal of presenting
Seneca music to the academic community and general public for the first time
accurately, sensitively, and on Seneca terms.
We envision a digital humanities project that presents high-quality videos of
the songs and dances with indigenous teaching and historical research about
the songs' origins and significance, as part of a co-authored, born-digital
book of public scholarship.
The book will focus on these social dance songs as a living tradition, drawing
on Bill Crouse's experetise as a practitioner of the oral tradition; and as as
historic one, drawing on my archival research in the papers of anthropologists
Lewis Henry Morgan (who first described the dances in English) and William
Fenton, and records of Seneca performances in the region through the
period of colonization.

The clearing at the woods' edge stands symbolically at the boundary between
indigenous traditional knowledge and modern experience under colonization,
between Seneca communities and Euro-American ones.
As Bill Crouse explains, the songs connect participants with Mother Earth
(\foreign{Ethinö’eh yöëdzade’}), uniting the community with all living and
non-living beings.
As an ancient oral tradition that practitioners are constantly finding new
ways to employ to meet present needs, the Earth Songs also connect history and
tradition, memory and creativity.
This project is the first academic study dedicated to Seneca Earth Songs, and
the first study of Seneca music and dance to focus on the perspectives of
indigenous practitioners.

\subsection{Relationship to Existing Studies}

This study aims to address a lack of trustworthy, in-depth resources for
non-indigenous people who are interested in Native American music, who want to
build a closer relationship to the Seneca people, and who are seeking a deeper
understanding of the role of indigenous people in American history.
Apart from presentations at places like Ganondagan, it is difficult for
scholars, teachers, and community members to know whether they can trust the
handful of online videos or the outdated scholarly publications.
The first detailed description of Seneca social dances was written by Lewis
Henry Morgan in his \wtitle{The League of the Haudenosaunee, or Iroquois},
published here in Rochester in 1851, and based on his first-hand observations
of Seneca practices with the help of Seneca informants
\Autocite{Morgan:League}.
He describes dances as so central to Haudenosaunee culture that \quoted{they
contain within themselves a picture and a realization of Indian life}, to the
extent that when the dance \quoted{loses its attractions, they will cease to
be Indians}
\Autocite[261, 263]{Morgan:League}.
He later argues, however, that Native Americans will only be able to survive
in \quoted{civilization} if they give up their traditional ways, and so his
book is at once a tribute to indigenous culture and an instrument of its
destruction.

After Morgan there only a few published studies that investigate Haudenosaunee
social songs, made by anthropologist William Fenton and dance ethnographer
Gertrude Kurath in the 1940s--60s
\Autocites{FentonKurath:EagleDance}{Kurath:IroquoisMusic}{Caldwell:Kurath}.
Kurath is the only source for detailed study of the actual sound and structure
of Haudenosaunee songs, but her taxonomies and speculative analysis are based
on criteria foreign to indigenous teachings; Fenton's work is considered by
many Seneca people to be both offensive and factually wrong
\Autocite{McCarthy:Iroquoianist}.
It is in large part because of the cavalier and inaccurate way with which
these researchers treated longhouse ceremony that those ceremonies are now
kept so strictly private.
In other words, the only available scholarly resources that specifically treat
Seneca social dance do so in the context of a large amount of information that
Seneca faithkeepers do not want the public to see or hear, partly for their
own protection.

The proposed project differs from these previous studies because it focuses on
music that Seneca people are actually willing to share, and moreover, on music
that Seneca people are currently using to build intercultural relationships.
This project will highlight the voice---both singing and teaching---of an
expert Seneca musician as well as perspectives from other musicians, and will
aim to present their views in their own terms without importing foreign
analytical categories or using their lived experience as raw material for
scholarly theorization in the exploitative fashion that has characterized
previous attempts.
This approach is in line with more recent approaches that model collaborative
work between non-Native scholars and Native experts, such as Beverley
Diamond's excellent but brief introduction to Haudenosaunee music 
\Autocite{Diamond:NativeAmericanNortheast}; 
and with a growing literature that emphasizes the modernity and creativity of
Native music as a contemporary practice
\Autocites
{Browner:FirstNations}
{Browner:Heartbeat}
{LevineRobinson:MusicModernity}.

\subsection{Benefit to Scholars and the Public}

This project will benefit humanities scholars who need reliable information
about Haudenosaunee traditions not only to understand indigenous values in
their own right but also to properly understand American history.
Too much past scholarship on Native music presented indigenous peoples in a
historical present tense: they told the story about \quoted{how Indians sing}
with the assumption that their singing like their broader identity is part
of the past and will eventually remain there.
Morgan argued that unlike the unrelentingly progressive and expansive aspect
of Western civilization, Indian culture was static and could never adapt to
the modern world.
Though Fenton and Kurath do not explicitly share Morgan's racialized ideology
of white supremacy, their accounts still treat present-day Native practice as
static and of primary interest for what it tells white researchers about the
past.
What is worse, no textbook history of Western music used in today's collegiate
curricula includes Native American music as part of America's story, past or
present, except perhaps for a brief mention in connection with Dvo\v{r}ak's
\wtitle{New World Symphony}.
Early-American music scholarship still rarely integrates indigenous knowledge,
which would have strengthened otherwise excellent recent historical studies of
colonial American music drawn from Euro-American documents
\Autocites{Goodman:IndianPsalmody}{Eyerly:Moravian}.
In fact even those documents reveal the constant presence of Native song in
the sound world of early America, from the earliest Jesuit Relations from
French Canada to the eighteenth-century accounts of English treaty proceedings
printed by Benjamin Franklin, from Morgan's observations in the 1840s through
annual public performances in the 1930s at Seneca encampments in Rochester's
Maplewood Park.
Seneca social songs have been performed in downtown Rochester for far longer
than Handel's \wtitle{Messiah} have been sung by the Rochester Oratorio
Society or Beethoven's symphonies have been played at the Eastman School of
Music; because, defying Morgan's prophecy, Seneca people have never stopped
living here on their ancestral lands, and they have never ceased dancing.

By collaboratively combining archive-based historical work and ethnographic
research, including looking together at archival sources with indigenous
experts, this project will provide a more balanced view of the sounds of early
America and the way music functioned in the intercultural dynamics that
shaped this country.
Moreover, compared to the staid rituals of a conservatory recital, in which
standard programs have changed very little in a century, a semi-annual
\quoted{Sing} event could include hundreds of newly composed songs (in the
\foreign{ehsgä:nye’} genre), not to mention musicians like the late Joanne
Shenandoah who fuse traditional and modern styles.

\subsection{Rationale for Digital Publication}

The interactive, interlinked nature of a website is well suited to the
relational and participatory character of the Earth Songs and the way they are
taught and shared in Seneca communities.
The digital format will allow the book-website to be freely and readily
accessible to a much wider audience than most academic music books reach, and
in an open format that is more amenable to input and modification from the
Seneca community and more in keeping with traditional Haudenosaunee values for
how to share and preserve cultural knowledge.
The project will demonstrate that one cannot understand the history of music
in America without knowing Haudenosaunee song, and that Native song is neither 
stuck in a primitive present tense nor lost to the past.
On the contrary, the website will highlight the vibrancy and power of
contemporary Seneca dancing and singing as a way of furthering the cause of
indigenous self-determination and cultural revitalization.
The recovery of descriptions, song lists, and recordings from the ethnographic
papers of Morgan and Fenton (among other archival sources) will constitute a
form of intellectual and cultural repatriation
\Autocite{Fox:Repatriation}.
% }}}1
% {{{1 organization, concepts
\clearpage
\section{Organization, Concepts, and Methods}

% {{{2 concepts
The key concepts in this project are three pairs of terms: Earth/land,
relationship/reciprocity, and tradition/history.
Seneca social songs both celebrate and enact a spiritual and ecological
relationship with the Earth, while also connecting Seneca people to the
specific land of their ancestry.
Filming the songs in significant places will boster the spiritual, moral, and
legal claims of the Seneca nation to their homelands (guaranteed to them by
the still-valid Candandaigua Treaty of 1794)
\Autocites{Deloria:BrokenTreaties}{BasicCall};
while the historical aspect of the project will firmly situate Seneca singing
as an ever-present part of the story of this region.

Relationship and reciprocity are widely acknowledged core values for Native
North Americans.
These concepts define the way the Haudenosaunee teach, learn, and perform
songs, and how they share them with outsiders.
The concept of the Covenant Chain recurs throughout colonial-era treaty
negotiations: an iron chain linked the first European ship that arrived on the
North Atlantic coast to the longhouse of the Haudenosaunee confederacy, and
both sides had an obligation to keep it polished and free from rust.
For me as an Indiana native descended from German settler-colonialists, this
project provides a way to take up the long-overdue work of polishing the
covenant chain, working toward restoration of mutually beneficial
relationships between indigenous and settler Americans. 

Finally, this project explores the complex relationship between history and
tradition in both indigenous and Western conceptions.
The project will demonstrate that one cannot understand the history of music
in America without knowing Haudenosaunee song, and that Native song is neither 
stuck in a primitive present tense nor lost to the past.
The goal, however, is not simply to fit Native music into a Western historical
framework; 
for indigenous North Americans, singing itself constitutes a form of
historical knowledge and provides its own ways of connecting past, present,
and future
\Autocite{Diamond:NativeAmericanHistory}.
% }}}2
% {{{2 components
\subsection{Components}

The project will be based on contemporary performances, interviews, and
fieldwork observations; archival recordings, chiefly at the American
Philosophical Society (APS) in Philadelphia and the Library of Congress; and
archival documents included those held at the University of Rochester (Lewis
Henry Morgan papers), the Rochester Museum and Science Center (Morgan and Ely
Parker collections), the APS (Fenton papers), the New York State Museum.

The website will feature new high-quality videos of Bill Crouse and others
singing Earth Songs in beautiful and significant outdoor locations
across ancestral Seneca territory, in different seasons appropriate to each
song.
For each type of song there will be a written introduction, video interviews
or stories about the structure and use of the song, storeis of how the songs
were learned and taught, and philosophical reflections on their relation to
Seneca traditional worldview and ethics.
The website will also provide users with information about issues of cultural
sensitivity, appropriation, and ethical use; and I will consult with
indigenous contributor to ensure that all materials are made available with
appropriate licenses
\Autocite{Christen:RelationshipsNotRecords}.
One of the chief benefits of the site to Seneca people, according to Bill
Crouse, would be to make accessible in one location a full library of
historic social-song recordings, including the ethnographers' field recordings
and those captured on cassette tape by participants in a Sing.

Though the precise relationship of book and website is still developing, the
book could include five chapters, with the first three based primarily on
interviews with Bill Crouse and the last two based more on my historical
research.
The first chapter will focus on the songs' relationship to the earth:
we will explore the geographical places and environments in which the songs
have been performed, and the traditional teachings about the spiritual
dimension of the songs in relation to the Earth.
The second chapter will go into detail about the musical structure and
patterns of these songs, not imposing European analytical categories but
helping readers understand indigenous ways of understanding singing and dance.
The third chapter will discuss teaching and tradition, tracing genealogies of
teachers and methods of oral transmission, and showing how the Seneca people
kept these traditions alive despite many obstacles, from land dispossession
and forced removal to boarding schools and the Kinzua dam tragedy, and how
younger generations of Onöndowa’ga:’ musicians are still responding creatively
to tradition and finding its relevance to their contemporary situation.
Chapter four will trace the origin and history of the songs, connecting
indigenous oral traditions with written archival documents that make it
possible to trace these songs back to the first encounters with Europeans.
Reading descriptions by Morgan and Fenton together with indigenous
practitioners may reveal previously unrecognized information about historic
Seneca music.
The final chapter will look at the earth songs within the context of a long
history of intercultural exchange, preceding European encounter and continuing
today, in which Haudenosaunee people have used songs at the woods' edge to
share their community with outsiders and build mutually beneficial
relationships.
% }}}2
% }}}1
% {{{1 competencies
\section{Competencies, Skills, and Access}

I began exploring this new research area with the support of a Humanities
Center Fellowship from my university in 2020, with the goal of understanding
patterns of intercultural exchange among different communities of colonial-era
New York.
This interest grew out of my previous studies of sacred song in early modern
Germany Lutheran and Spanish Catholic communities, which led me into colonial
and decolonial studies.
While funding and travel were limited during the pandemic I did extensive
background reading in Native American studies, history, and ethnomusicology,
and began research in UR's own archive.
Since inviting Bill Crouse to teach in my Music History classroom in 2021 we
have been working together to develop a project that would benefit the Seneca
community.

In this project I will draw on my experience in archive-based historical
research while extending into the new research methods of ethnographic
fieldwork.
I draw on graduate training in ethnomusicology and ritual studies, but I have
also sought out mentorship from ethnomusicologist Ellen Koskoff, and am
building a network of indigenous advisers inside and outside the academy.
As a professional musician I have ample experience learning and transcribing
traditional songs by ear.
I have developed an intermediate novice ability in the Seneca language,
studying for the past two years with Ja:no’s Bowen of the Seneca Nation
Language Department.
There are fewere than thirty fluent speakers and the Earth Songs mostly use
worldless vocables, so fluency is not necessary for this project.

My experience with digital humanities technologies prepares to ensure the
highest standards of typography and design, and to build an efficient and
stable software system.
I worked for seven years as a copy editor, I typeset and published two volumes
of critical music editions in \LaTeX{} and Lilypond, and I preparing the full
text of my monograph in \LaTeX{} with all diagrams, tables, and music
examples.
I have designed and maintained websites for research projects
(\url{www.arca1650.info}) and teaching (\url{www.andrewcashner.com}) since
2012.
My institution has provided \$5,000 of research funding that I can use as
start-up funding, and I will draw on the expertise of faculty and students in
audio-music engineering and computer-science departments.
% }}}2
% }}}1
% {{{1 final product
\section{Final Product and Dissemination}

The final product will be a born-digital book and website
(\url{http://www.senecasongs.info}); determining the optimal presentation
format for the subject matter will be one of the goals of the project.
The text of the book will be the core of the website, together with many of
the multimedia sources for the book.
Readers may experience the book as an online, potentially non-linear,
experience, or as a more traditional book, including the ability to download
the book as a PDF or order a print-on-demand copy.
The book and website will be generated from the same source texts, using a
simple, sustainable workflow based on free and open-source technology.
The website will be built on the core standards of the Internet (HTML5 and
CSS3) that are supported in every browser, minimizing scripts and plugins that
can break or become obsolete; the print form of the book will be typeset with
\LaTeX{}, a robust system with strong backwards-compatibility.
Through adaptive design the website will be equally accessible on desktop and
mobile devices, screen readers and text-based browsers, and will be fully
ADA-compliant.

I will be responsible for maintaining the website, including providing 
funding for hosting, but I will also seek out a Seneca webmaster and provide
ways for the community to update or correct the site. 
Using open-source software not only makes the publication easier to update and
mantain, but it is essential when serving an economically
under-privileged community.
These tools provide the \quoted{liberation technologies} demanded by
Haudenosaunee activists and make it possible for Native nations to exercise
their rights, validated by the UN Declaration on the Rights of Indigenous
Peoples, to control their own representation and cultural heritage
\autocite{BasicCall}.

Given the collaborative nature of a project focused on protected indigenous
cultural heritage, the review process must necessarily be distinct from the
traditional academic model.
I plan to assemble a panel of consultants, including both indigenous academics
and non-academic experts.
After the their own initial peer review, the panel will be empowered to seek
out additional, potentially anonymous reviews.
I will hire proficient Seneca-language speakers to review the linguistic
elements and I will invite Seneca community members to test the site.
% }}}1
\end{document}
