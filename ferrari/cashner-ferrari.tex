\documentclass[12pt]{article}
\usepackage[T1]{fontenc}
\usepackage{ebgaramond}
\usepackage[margin=1.25in]{geometry}
\usepackage{semantic-markup}
\usepackage{microtype}
\frenchspacing
\usepackage[all]{nowidow}
\usepackage[notes]{biblatex-chicago}
\addbibresource{master.bib}

\title{Ferrari Award Proposal}
\author{Andrew A. Cashner}

\begin{document}
\maketitle

I am seeking funding for a research project entitled \wtitle{Songs at the
Woods’ Edge: The Earth Songs of the Seneca Nation}.
This will be a collaborative project with Bill Crouse, Sr., the renowned
Seneca master musician and faithkeeper, focused on the traditional repertoire
of Onöndowa'ga:' social dance songs, also known as earth songs.
We envision two main outputs: (1) a co-authored book of public scholarship and
(2) a website intended as a resource for the public and especially for
educators, featuring high-quality videos of performances of the songs and
dances and teaching about them, presented from an indigenous perspective and
according to indigenous values.
The book will focus on these social dance songs as a living tradition, drawing
on Bill Crouse’s expertise as a practitioner of the oral tradition; and as a
historic one, drawing on my archival research in Lewis Henry Morgan’s papers
and other sources that document these songs in the earliest intercultural
encounters. 

The central concept of the book comes from the historic Seneca practice of the
\quoted{Woods'-Edge Greeting}, which was a way of welcoming outsiders into the
village and played an important role both in the functioning of the
Haudenosaunee (Iroquois) League government and in treaty negotiations with
Euro-Americans.
Seneca musicians are still using these songs to create a meeting space where
Native and non-Native people, and traditional and non-traditional Seneca
people, can develop reciprocal relationships.
In fact, the entryway to Ganondagan, the Seneca Cultural Center near Victor,
is designed around the traditional Woods'-Edge Greeting, and Bill Crouse
regularly presents social songs there in events that are free and open to the
public, using the songs as a way to teach traditional Onöndowa'ga values and
to highlight the resiliency and vibrancy of Seneca culture today.

We envision five chapters for the book, with the first three based primarily
on interviews with Bill Crouse and the last two based more on my historical
research.
The first chapter will focus on the songs' relationship to the earth: this
includes both the geographical places and environments in which the songs are
performed today, and where they were sung across the ancestral Onöndowa'ga:'
territory currently occupied by western New York; and it includes the
traditional teachings about the spiritual dimension of the songs as means of
connecting people to the earth and to each other.
The second chapter will go into detail about the musical structure and
patterns of these songs, not imposing Western analytical categories but
helping readers understand indigenous ways of hearing and thinking about
singing and dance.
The third chapter will discuss teaching and tradition, tracing genealogies of
teachers and methods of oral transmission, and showing how the Seneca people
kept these traditions alive despite many obstacles, from land dispossession
and forced removal to boarding schools and the Kinzua dam tragedy.
Moreover it will show how the younger generations of Seneca musicians are
still responding creatively to these traditions and finding its relevance to
their contemporary situation.

Chapter four will trace the origin and history of the songs, engaging with
written archival documents that document these songs in the earliest
Euro-American encounters.
This will correct errors and fill in gaps in the small body of existing
scholarship, and draw on previously unrecognized archival sources, seen
through indigenous eyes.
The final chapter will look at the earth songs within the context of a long
history of intercultural exchange, preceding European encounter and continuing
today.

This project would meet a significant need for trustworthy information about
Native American music, provided with the authority and blessing of indigenous
experts and their communities.
The few works of published scholarship all have serious problems.
The first detailed description of these dances was written by Lewis Henry
Morgan in his \wtitle{The League of the Haudenosaunee, or Iroquois}, written
here in Rochester in 1851, and based on his first-hand observations of Seneca
practices and information from his mostly Seneca informants.%
\Autocite{Morgan:League}
He describes dances as so central to Haudenosaunee culture that \quoted{they
contain within themselves a picture and a realization of Indian life}, to the
extent that when the dance \quoted{loses its attractions, they will cease to
be Indians} (263).%
\Autocite[261, 263]{Morgan:League}
He later argues that Native Americans will only be able to survive in
\quoted{civilization} if they give up their traditional ways, and so his book
is at once a tribute to indigenous culture and an instrument of its
destruction.
Morgan's papers are preserved primarily here at the University of Rochester,
and likely contain much information that did not make it into his published
work.
Like many Euro-American observers before him, he observed many practices that
he did not understand, but his descriptions may yield intelligible information
to contemporary pracitioners like Bill Crouse.

In the twentieth century, William Fenton and Gertrude Kurath published
ethnographic studies of Haudenosaunee dances and ceremonial songs which,
though based on first-hand observation, were not sufficiently informed by
insider perspectives or attuned to the priorities and values of the people
they were observing.%
\Autocites{Fenton:GreatLaw}{Kurath:IroquoisMusic}
Kurath is the only source for detailed study of the actual sound and structure
of Haudenosaunee songs, but her criteria for analysis are not at all based on
indigenous teachings, and much of her work is purely speculative.
In Fenton's case, his work is actually considered by many Seneca people to be
both offensive and factually wrong.
He made ceremonial information public that was meant to be kept private, even
within Haudenosaunee communities.
All the same, in a foolhardy attempt to reconstruct the original ceremonies by
which the Haudenosaunee League constituted itself, Fenton identified a large
number of historical and archival sources that describe songs and dances still
practiced today.
As with Morgan, Fenton gathered much valuable information even if he made
questionable use of it.
Scrutiny of his published and manuscript writings together with an indigenous
expert is likely to yield information about Seneca traditions that may have
been lost over the years, or may simply make it possible to correct the
published scholarly record.

I began this project with the support of a Humanities Center Fellowship in
spring 2020. 
Since then I have been building a network of connections in the Seneca Nation 
and learning the Seneca language with Ja:no's Bowen of the Allegany Territory
Language Department, while I have done most of the necessary background
reading in primary and secondary sources and have begun working with archival
sources in our own Rare Books and Special Collections relative to local Native
American history.
This work extends my work on music and religious belief in colonial Latin
America into a new part of the world, requiring a shift in methodology and a
new ethical sensitivity with respect to indigenous peoples.%
\begin{Footnote}
    \Autocites{Cashner:HearingFaith}
    {Cashner:Cards}
    {Cashner:ImitatingAfricans};
    \wtitle{Villancicos about Music from Seventeenth-Century Spain and New
    Spain}, vols. 1--2, Web Library of Seventeenth-Century Music no. 32, 36
    (Society for Seventeenth-Century Music, 2017, 2021).
\end{Footnote}
I have incorporated what I have been learning into my teaching, as I have
included units on Haudenosaunee song in courses on music appreciation and
music history, including hosting Bill Crouse as a guest teacher in spring 2021
and again this spring.

I would use the money to pay Mr. Crouse to film a series of interviews and
performance videos at significant sites around ancestral Seneca territory, to
pay for audio and visual technicians and equipment needed to present the songs
in the most appealing and engaging way we can, and to pay for web hosting
and publication costs. 
I hope to draw on the resources of our Department of Audio and Music
Engineering to produce the media, and I can envision fruitful collaborations
with our programs in dance and theater and events through the Humanities
Center to present our findings to our academic community and to the public.
I will seek connections with Ganondagan and the Seneca-Iroquois National
Museum in Salamanca.

I intend to do most of the filming in the summer and work on producing the
videos and writing the book in academic year 2022--2023, with the goal of a
finished product by summer 2023.

\end{document}
