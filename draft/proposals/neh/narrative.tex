\documentclass{neh}
\title{Narrative}
\begin{document}
\maketitle
% {{{1 significance
\section{Significance and Contribution}
% {{{2 overview
\wtitle{The Earth Songs of the Seneca Nation} is a digital-humanities project
on the subject of the traditional social-dance songs of the Onöndowa’ga:’
people, created in collaboration with Seneca singer and faithkeeper Bill
Crouse, Sr.
The original inhabitants of the land now occupied by western New York, the
Senecas are one of the Six Nations of the Haudenosaunee (Iroquois)
Confederacy.
Traditional Seneca music is primarily vocal, supported by water drum and
rattle, and usually combined with dance; the songs are divided into ceremonial
and social functions.
Ceremonial songs hold sacred power as part of longhouse ceremonies including
healing rituals; they are closed to non-Senecas and many are even kept private
within the Seneca community, reserved only for those who need them.
Social songs, by contrast, are shared openly.

Known as Earth Songs (\foreign{yöëdza’ge:ka:’ gaë:nö’shö’}), these songs have
been used for centuries to build reciprocal relationships within the Seneca
community and with outsiders.
The earliest European interlopers in Seneca country report being greeted at
the woods' edge with songs.
At Ganondagan, the Seneca Arts and Culture Center near Victor, New York,
visitors pass through an entryway designed around the traditional woods'-edge
greeting to hear regular presentations of Earth Songs by Seneca singers like
Bill Crouse.
These presentations create a space like the woods'-edge clearing of earlier
days in which to share Seneca teachings and values with outsiders.
To sing at the woods' edge means to stand at the boundary between indigenous
traditional knowledge and modern experience under colonization, and between
Seneca communities and Euro-American ones.
As an ancient oral tradition that practitioners are constantly finding new
ways to employ to meet present needs, the Earth Songs sung in that space
connect history and tradition, memory and creativity.

Working together with Bill Crouse and other Seneca practitioners, this project
will present Seneca Earth Songs to the academic community and general public
for the first time accurately, sensitively, and on Seneca terms.
Through a website and digital book of public scholarship, the project will
present high-quality videos of the songs and dances with information about
the songs' origins, structure, and significance.
It will draw on Bill Crouse's expertise as a practitioner of the oral
tradition, and my archival research into historic accounts of Seneca dance
from the Jesuit Relations through Lewis Henry Morgan and William Fenton.
% }}}2
% {{{2 relationship to existing studies
% \subsection{Relationship to Existing Studies}

This study aims to address a lack of trustworthy, in-depth resources for
learning about this type of Native American music.
According to Peter Jemison, recently retired director of Ganondagan, the
cultural center staff were flooded in the last two years by requests from
educators for information on how to include Native culture in their curricula.
For Haudenosaunee and Seneca music, though, there are few reliable sources
available.
In the first scholarly description of Haudenosaunee dance (based on Seneca
sources), Lewis Henry Morgan wrote in 1851 that the dances \quoted{contain
within themselves a picture and a realization of Indian life}, to the extent
that when the dance \quoted{loses its attractions, they will cease to be
Indians} 
\Autocite[261, 263]{Morgan:League}.
Morgan viewed the dances as static relics of a traditional past that Native
people would have to surrender in the face of progressive
\quoted{civilization}.
Twentieth-century ethnographers William Fenton and Gertrude Kurath made no
distinction between privileged ceremonial songs and social-dance songs, and as
a result their books are full of information that Seneca faithkeepers today do
not want to share with the public---not to mention the the inaccuracies and
non-indigenous categories of their analyses
\Autocites{Fenton:GreatLaw}{FentonKurath:EagleDance}{Kurath:IroquoisMusic}
{Caldwell:Kurath}{McCarthy:Iroquoianist}.

The proposed project differs from these previous studies because it focuses on
music that Seneca people are actually willing to share, and builds on the way
they are already using this music to build intercultural relationships.
The methodology follows the model of recent collaborative work between
non-Native scholars and Native experts, such as Beverley Diamond's excellent
though brief introduction to Haudenosaunee music 
\Autocite{Diamond:NativeAmericanNortheast},
and with a growing literature that emphasizes the modernity and creativity of
Native music as a contemporary practice
\Autocites
{Browner:FirstNations}
{Browner:Heartbeat}
{LevineRobinson:MusicModernity}.
This project has a more historical focus than those, however, as it combines
ethnographic fieldwork with archival research, including seeking out
indigenous perspectives on the archival documents.
Even the best historical studies of early American music that focus on
interactions with Native people are based on on Euro-American documents and
do not incorporate traditional indigenous knowledge and oral tradition 
\Autocites{Goodman:IndianPsalmody}{Eyerly:Moravian}.
No history of American music can claim coherence without including the music
of indigenous Americans, and no attempt at inclusion can succeed without
the collaboration of practitioners of the oral traditions.
% }}}2
% {{{2 benefit to audience
% \subsection{Benefit to Scholars and the Public}

This project will benefit humanities scholars, educators, and members of the
public by providing them with reliable information on Native American
music.
The knowledge shared through this project will help all of us to gain a deeper
understanding of the land we share.
Some indigenous people may deepen their connection to their own traditions;
non-indigenous people will be better equipped to build relationships with
Native American communities.
The interlinked nature of a website is well suited to the relational and
participatory character of the Earth Songs and the way they are shared in
Seneca communities.
The digital format will allow the book/website to be freely 
accessible to a wide public audience.
% }}}2
% }}}1
% {{{1 organization, concepts
\section{Organization, Concepts, and Methods}
% {{{2 concepts
The key concepts in this project are three pairs of terms: Earth/land,
relationship/reciprocity, and tradition/history.
Seneca social songs celebrate and enact a relationship with the Earth in both
ecological and spiritual terms, while also connecting Seneca people to the
land of their ancestry
\Autocites{Deloria:BrokenTreaties}{BasicCall}.
Relationship and reciprocity are widely acknowledged core values for Native
North Americans, and they define the way Haudenosaunee people teach and
present songs.
The concept of the Covenant Chain---linking the first European ship to the
Haudenosaunee longhouse---recurs throughout colonial treaty negotiations.
Both sides had an obligation to keep it free from rust.
For me as an Indiana native descended from German settler-colonialists, this
project provides a way to take up the long-overdue work of polishing the
chain of friendship, working toward restoration of mutually beneficial
relationships between indigenous and settler Americans. 
Exploring the complex relationship between history and tradition in both
indigenous and Western conceptions, this project will demonstrate that 
Native song is neither stuck in a primitive present tense nor lost to the past.
At the same time, the goal is not simply to fit Native music into a Western
historical framework; for indigenous North Americans, singing itself
constitutes a form of historical knowledge and provides its own ways of
connecting past, present, and future
\Autocite{Diamond:NativeAmericanHistory}.
% }}}2
% {{{2 components
% \subsection{Components}

The website will feature new high-quality videos of Bill Crouse and others
singing Earth Songs in beautiful and significant outdoor locations
across ancestral Seneca territory.
For each type of song there will be a written introduction, video interviews
or stories about structure and use of the songs, and philosophical reflections
on their relation to Seneca worldview.
The website will also provide users with information about issues of cultural
sensitivity, appropriation, and ethical use; and I will consult with
indigenous contributors to ensure that all materials are made available with
appropriate licenses
\Autocite{Christen:RelationshipsNotRecords}.
Sources include contemporary performances, interviews, and fieldwork
observations; ethnographic recordings at the American Philosophical Society
(APS) in Philadelphia and the Library of Congress; and archival
documents at the University of Rochester (Lewis Henry Morgan papers), the
Rochester Museum and Science Center (Morgan and Ely Parker collections), the
APS (Fenton papers), and the New York State Museum.
One of the chief benefits of the site to Seneca people, according to Bill
Crouse, would be to make accessible in one location a full library of
historic recordings, effectively repatriating the ethnographers' materials 
\Autocite{Fox:Repatriation}.

The book will include five chapters, with the first three based primarily on
interviews with Bill Crouse and the last two based more on my historical
research.
The first chapter will focus on the songs' relationship to the earth and the
land; the second will explore the musical structure and patterns of these
songs, emphasizing Seneca understandings of music.
The third chapter will trace genealogies of teaching and methods of oral
transmission, showing how the Seneca people kept their songs alive in defiance
of land dispossession, boarding schools, and the Kinzua dam tragedy, and how
younger generations are still responding creatively to tradition including
through the COVID-19 pandemic.
Chapter four will trace the origin and history of the songs, connecting
indigenous oral traditions with written archival documents from Euro-American
perspectives, reading Morgan and Fenton's field notes together with indigenous
practitioners.
The final chapter will look at the earth songs within the context of a long
history of intercultural exchange, preceding European encounter and continuing
today, in which Haudenosaunee people have used songs at the woods' edge to
share their community with outsiders and build mutually beneficial
relationships.
% }}}2
% }}}1
% {{{1 competencies
\section{Competencies, Skills, and Access}

I began exploring this new research area with the support of a Humanities
Center Fellowship from my university in 2020, with the goal of understanding
patterns of intercultural exchange among different communities of colonial-era
New York.
This interest grew out of my previous studies of sacred song in early modern
Germany Lutheran and Spanish Catholic communities, which led me into colonial
and decolonial studies.
In this project I will draw on my experience in archive-based historical
research while extending into the new research methods of ethnographic
fieldwork.
My graduate studies did include ethnomusicology and ritual studies, and I have
sought out mentorship from ethnomusicologist Ellen Koskoff.
As a professional church musician I have ample experience learning traditional
songs by ear.
I have attained intermediate novice ability in the Seneca language, studying
with Ja:no’s Bowen of the Seneca Nation Language Department.

My experience with digital humanities technologies prepares me to ensure the
highest standards of typography and design in a robust software system.
I worked for seven years as a copy editor, I typeset two published volumes
of critical music editions in \LaTeX{} and Lilypond, and I created all the
diagrams, tables, and music examples for my monograph.
I have designed and maintained websites for research projects and teaching
since 2012.
From my institution I have access to \$5,000 of research funding for start-up
funding, and I hope to collaborate with our Audio Music Engineering and
Computer Science departments.
% }}}2
% }}}1
% {{{1 final product
\section{Final Product and Dissemination}

The final product will a born-digital book and website
(\url{http://www.senecasongs.earth}).
Readers may experience the book as an online, non-linear, experience or as a
more traditional book, downloadable as a PDF.
The book and website will be generated from the same sources, using a
sustainable workflow based on free and open-source technology.
The website will be built on the core standards supported in every browser
(HTML5 and CSS3), minimizing scripts and plugins that can break or become
obsolete; the print form of the book will be typeset with the stable 
\LaTeX{} document-preparation system.
Through adaptive design the site will be equally accessible and ADA-compliant
on desktop and mobile devices and via screen readers.
I will host and maintain the website, but I will also seek out a Seneca
webmaster and provide channels for community feedback.
Using open-source software is essential when serving an economically
under-privileged community, and provides the \quoted{liberation technologies}
that Haudenosaunee activists have demanded to give them control
over their own representation
\autocite[123]{BasicCall}.

Given the collaborative nature of a project focused on protected indigenous
cultural heritage, the review process must necessarily be distinct from the
traditional academic model.
I plan to assemble a panel of consultants, including both indigenous academics
and non-academic experts.
After their own initial peer review, the panel will be empowered to seek
out additional, potentially anonymous reviews.
I will hire proficient Seneca-language speakers to review the linguistic
elements and I will invite Seneca community members to test the site.
% }}}1
\end{document}
